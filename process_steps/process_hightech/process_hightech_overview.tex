\tikzstyle{block} = [rectangle, draw, fill=blue!20, text width=3cm, text centered, rounded corners, minimum height=1.5cm]
\tikzstyle{line} = [draw, very thick, color=black!50, -latex']

The general flow chart of the overall process flow can be seen in \autoref{full_flow}.
These process steps will be discussed within the following sections.
\begin{figure}[H]
	\centering
	\begin{tikzpicture}[node distance=2cm, thick,scale=0.8, every node/.style={transform shape}]
		%% Place nodes
		%active CMOS	
		% first row

		\node [block]  (sti)  at (0,16)  {Isolation (STI)\\ \autoref{sti_chapter}};
		\node [block, below of=sti] (nwell) {N-Well\\ \autoref{nwell_chapter}};
		\node [block, below of=nwell] (pwell) {P-Well\\ \autoref{pwell_chapter}};
		\node [block, below of=pwell] (pbase) {P-Base\\ \autoref{pbase_chapter}};
		\node [block, below of=pbase] (nbase) {N-Base\\ \autoref{nbase_chapter}};		;

		%second row
		\node [block, right=1cm of nbase] (fox) {Field oxide\\ \autoref{fox_chapter}};
		\node [block, above of=fox] (gate) {Gate\\ \autoref{gate}};
		\node [block, above of=gate] (np) {n+ Implant\\ \autoref{nimplant}};
		\node [block, above of=np] (pp) {p+ Implant\\ \autoref{pimplant}};
		\node [block, above of=pp] (silicification){Silicification\\ \autoref{step_silicification}};

		%post proces
		\node [block, right=1cm of silicification] (contact) {Contact \\ \autoref{contact}};
		\node [block, below of=contact] (metal) {First metal\\ \autoref{metal}};
		\node [block, below of=metal] (via) {Additional vias\\ \autoref{via}};
		\node [block, below of=via] (more_metal) {Additional metal\\ \autoref{more_metal}};
		%\node (repeat) at (10.5,14.5) {Repeat};

		%% Draw edges
		\path [line] (sti) -- (nwell);
		\path [line] (nwell) -- (pwell);
		\path [line] (pwell) -- (pbase);
		\path [line] (pbase) -- (nbase);
		\path [line] (nbase) -- (fox);
		\path [line] (fox) -- (gate);
		\path [line] (gate) -- (np);
		\path [line] (np) -- (pp);
		\path [line] (pp) -- (silicification);
		\path [line] (silicification) -- (contact);
		\path [line] (contact) -- (metal);
		%\path [line] (metal) -- (via);
		%\path [line] (via) -- (more_metal);
		%\path [line] (more_metal) -- +(3,0) -- +(3,-2) -- (via);

		\draw[dotted] (-2,6) rectangle (6.25,17);
		\node at (2,6.5) {CMOS process};
		\draw[dotted] (6.5,6) rectangle (12,17);
		\node at (8,6.5) {Interconnect};

		%\draw[dotted] (1.5,9) rectangle (10.5,21.5);
		%\node at (4,9.5) {Front-end processing};

		%\draw[dotted] (11,9) rectangle (15,21.5);
		%\node at (13,9.5) {Back-end processing};
	\end{tikzpicture}
	\caption{Frontend and backend process flow}
	\label{full_flow}
\end{figure}
The six overall process steps are part of an active part of the technology, while the final metal (respectively contact) layers will be used for making a contact between the logic gates and macro cells and making them available to the exterior world.

For this process p-substrate is the required basic substrate, but forks and modifications will be very well possible based on a Graphene substrate or alike, still under the LSPL.
The starting material is a p-type, <100> oriented silicon with a doping concentration of $\approx 9\times10^{14}cm^{-3}$.\\

\textbf{Machines required}:
\begin{itemize}
	\item Ion implanter
	\item Plasma etcher
	\item Sputter engine (Metal deposition) 
	\item Diffusion furnace
	\item Exposure unit
\end{itemize}
