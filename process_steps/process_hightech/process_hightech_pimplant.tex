\subsection{p+ Implant}\label{pimplant_chapter}
For the bulk of the NMOS transistors and for the source and drain of the PMOS transistors highly doped  p+ areas are required.
In this step we're going to build these.

\begin{figure}[H]
	\centering
	\begin{tikzpicture}[node distance = 3cm, auto, thick,scale=\CrossAndTopSectionBig, every node/.style={transform shape}]
		\input{tikz_process_steps/fox.a.tex}

\paintnwells{3.5}
\paintpwells{3.5}
\paintpbases{2.0}
\paintnbases{1.5}

\fill[gateoxide] (5.00,\STIIslandSurface) rectangle (6.00,\gateoxidetop);
\fill[poly] (5.00,\gateoxidetop) rectangle (6.00,\polytop);

\fill[gateoxide] (12.00,\STIIslandSurface) rectangle (13.00,\gateoxidetop);
\fill[poly] (12.00,\gateoxidetop) rectangle (13.00,\polytop);

\fill[gateoxide] (21.90,\STIIslandSurface) rectangle (22.90,\SONOStopONE);
\fill[nitride] (21.90,\SONOStopONE) rectangle (22.90,\SONOStopTWO);
\fill[gateoxide] (21.90,\SONOStopTWO) rectangle (22.90,\SONOStopTHREE);
\fill[poly] (21.90,\SONOStopTHREE) rectangle (22.90,\polytop+0.2);

% poly diode
\fill[poly] (43.00,\STIIslandSurface+0.75) rectangle (48.00,\polytop+0.75);

% poly resistor 
\fill[poly] (48.50,\STIIslandSurface+0.75) rectangle (54.50,\polytop+0.75);

\fill[isolationoxide] (45.0,\polytop+0.75) rectangle (46.0,\implantstoptop);


\paintnimplants{0.5}
\paintpimplants{0.5}


	\end{tikzpicture}
	\begin{tikzpicture}[node distance = 3cm, auto, thick,scale=\CrossAndTopSectionBig, every node/.style={transform shape}]
		\input{tikz_process_steps/pimplant.b.tex}
	\end{tikzpicture}
	\caption{P+ implant geometry target}
\end{figure}

The tricky thing here is to have a reasonable implant depth but not too deep because the deeper the junction, the higher the junction capacity which in turn limits the switching performance of the CMOS circuitry.

\begin{figure}[H]
	\centering
	\begin{tikzpicture}[node distance =1cm, auto, thick,scale=\VLSILayout, every node/.style={transform shape}]
		\input{tikz_process_steps/pimplant.layout.tex}
	\end{tikzpicture}
	\caption{P+ layout}
	\label{pimplant_layout}
\end{figure}

An example layout of p-implants can be seen in \autoref{pimplant_layout}, the mask is being extracted from the layer "p\_plus\_select"

Also important to notice is that this example layout is just for demonstration purposes only, please have a look at the standard cell documentation for the actual layouts. 

The pselect is implanted with a Boron ($B^{11}$) dose of $2.5\times10^{12}cm^{-2}$ at an energy of 20 keV  (43nm$\pm$18nm deep)

\newpage
