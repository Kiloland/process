\subsection{p+ Implant}\label{pimplant_chapter}

For the bulk of the NMOS transistors and for the source and drain of the PMOS transistors highly doped  p+ areas are required.

In this step we're going to build these.

\begin{figure}[H]
	\centering
	\begin{tikzpicture}[node distance = 3cm, auto, thick,scale=\CrossAndTopSectionBig, every node/.style={transform shape}]
		\input{tikz_process_steps/fox.a.tex}

\paintnwells{3.5}
\paintpwells{3.5}
\paintpbases{2.0}
\paintnbases{1.5}

\fill[gateoxide] (5.00,\STIIslandSurface) rectangle (6.00,\gateoxidetop);
\fill[poly] (5.00,\gateoxidetop) rectangle (6.00,\polytop);

\fill[gateoxide] (12.00,\STIIslandSurface) rectangle (13.00,\gateoxidetop);
\fill[poly] (12.00,\gateoxidetop) rectangle (13.00,\polytop);

\fill[gateoxide] (21.90,\STIIslandSurface) rectangle (22.90,\SONOStopONE);
\fill[nitride] (21.90,\SONOStopONE) rectangle (22.90,\SONOStopTWO);
\fill[gateoxide] (21.90,\SONOStopTWO) rectangle (22.90,\SONOStopTHREE);
\fill[poly] (21.90,\SONOStopTHREE) rectangle (22.90,\polytop+0.2);

% poly diode
\fill[poly] (43.00,\STIIslandSurface+0.75) rectangle (48.00,\polytop+0.75);

% poly resistor 
\fill[poly] (48.50,\STIIslandSurface+0.75) rectangle (54.50,\polytop+0.75);

\fill[isolationoxide] (45.0,\polytop+0.75) rectangle (46.0,\implantstoptop);


\paintnimplants{0.5}
\paintpimplants{0.5}


	\end{tikzpicture}
	\caption{P+ implant geometry target}
\end{figure}

The tricky thing here is to have a reasonable implant depth but not too deep because the deeper the junction, the higher the junction capacity which in turn limits the switching performance of the CMOS circuitry.

Also important to notice is that the implantation energy must not be too high, otherwise the dopants may leak through the polysilicon gate.

The pselect is implanted with a Boron ($B^{11}$) dose of $2.5\times10^{12}cm^{-2}$ at an energy of 20 keV  (43nm$\pm$18nm deep)

