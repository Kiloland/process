\section{Shallow trench isolation}\label{sti_chapter}
The geometry of a substrate with STI implemented can be seen in \autoref{sti_target}.

\begin{figure}[H]
	\centering
	\begin{tikzpicture}[node distance = 3cm, auto, thick,scale=\CrossAndTopSectionBig, every node/.style={transform shape}]
		%silicon oxide
\fill[isolationoxide] (0,0) rectangle (55,\STIIslandSurface);

% substrate
\fill[substrate] (0,0) rectangle (55,\trenchBottom);
\node at (2,0.5) {Silicon substrate};

% normal wells
\fill[substrate] (1.25,\trenchBottom) rectangle (8.25,\STIIslandSurface);
\fill[substrate] (9.75,\trenchBottom) rectangle (16.75,\STIIslandSurface);
\fill[substrate] (18.25,\trenchBottom) rectangle (25.25,\STIIslandSurface);
\fill[substrate] (26.75,\trenchBottom) rectangle (33.75,\STIIslandSurface);
\fill[substrate] (35.25,\trenchBottom) rectangle (42.25,\STIIslandSurface);



	\end{tikzpicture}
	\begin{tikzpicture}[node distance = 3cm, auto, thick,scale=\CrossAndTopSectionBig, every node/.style={transform shape}]
		% trench area
\fill[isolationoxide] (0,0) rectangle (55,9);

\fill[substrate] (1.25,1) rectangle (8.25,7.25);
\fill[substrate] (9.75,1) rectangle (16.75,7.25);
\fill[substrate] (18.25,1) rectangle (25.25,7.25);
\fill[substrate] (26.75,1) rectangle (33.75,7.25);
\fill[substrate] (35.25,1) rectangle (42.25,7.25);
	\end{tikzpicture}
	\caption{Shallow trench isolation target geometry}
	\label{sti_target}
\end{figure}

As can be seen in \autoref{tripple_well_target}, the N-well and the STI trench are supposed to have approximately the same depth but the N-well and P-well go down a little bit further.

Because the N-well will be $\approx 4 \mu m$ in depth we have to match this with our trench depth.

I order to allow a sufficiently low resistance of the ESD diode but at the same time a sufficient isolation of between the standard cells a trade-off has been done.

The targeted depth of the box isolation is $\approx 4 \mu m$.

\begin{figure}[H]
	\centering
	\begin{tikzpicture}[node distance =1cm, auto, thick,scale=\VLSILayout, every node/.style={transform shape}]
		\input{tikz_process_steps/sti.layout.tex}
	\end{tikzpicture}
	\caption{Shallow trench isolation layout}
	\label{sti_layout}
\end{figure}

In \autoref{sti_layout} we can see the layout for the STI area.

The STI area will be everywhere, where no well areas are.

The field oxide needs to be grown out of trenches which can't be etched out of the silicon by using resist as a mask.

For that reason we will have to resort to a protective mask made from a silicon dioxide layer which has to be etched before hand.

So the mask will be exposed onto positive resist on top of the hard mask oxide layer in order to form a protective mask covering the active areas from having etched trenches into them.

After that we can either use a dry etching method or wet etching for cutting into the silicon substrate and making the active area become islands with trenches in between.

After these steps we have to remove the hard mask.

Our minimum width and height as well as the space between the active areas comes from the line space constrain of the silicon etcher and of course the optical limitations of the stepper which are as well 0.5\um.

\newpage

\subsection{Silicon etching}\label{sti_trench_etch}

The trench depth has to be at least 4.5 microns deep, in order to compensate for sacrificial silicon polishing during the later CMP steps.

\begin{figure}[H]
	\centering
	\begin{tikzpicture}[node distance = 3cm, auto, thick,scale=\CrossSectionOnly, every node/.style={transform shape}]
		\fill[substrate] (0,0) rectangle (55,\STIIslandSurface);
\node at (2,0.5) {Silicon substrate};

\fill[isolationoxide] (0,\STIIslandSurface) rectangle (55,\STIIslandSurface+0.25);
\fill[nitride] (0,\STIIslandSurface+0.25) rectangle (55,\STIIslandSurface+1.0);


\fill[resist] ( 1.25,\STIIslandSurface+1.0) rectangle ( 8.25,\STIIslandSurface+3.0);
\fill[resist] ( 9.75,\STIIslandSurface+1.0) rectangle (16.75,\STIIslandSurface+3.0);
\fill[resist] (18.25,\STIIslandSurface+1.0) rectangle (25.25,\STIIslandSurface+3.0);
\fill[resist] (26.75,\STIIslandSurface+1.0) rectangle (33.75,\STIIslandSurface+3.0);
\fill[resist] (35.25,\STIIslandSurface+1.0) rectangle (42.25,\STIIslandSurface+3.0);

	\end{tikzpicture}
	\drawStepArrow{}
	\begin{tikzpicture}[node distance = 3cm, auto, thick,scale=\CrossSectionOnly, every node/.style={transform shape}]
		\fill[resist] (1.25,\trenchBottom) rectangle (8.25,\STIIslandSurface+2.0);
\fill[resist] (9.75,\trenchBottom) rectangle (16.75,\STIIslandSurface+2.0);
\fill[resist] (18.25,\trenchBottom) rectangle (25.25,\STIIslandSurface+2.0);
\fill[resist] (26.75,\trenchBottom) rectangle (33.75,\STIIslandSurface+2.0);
\fill[resist] (35.25,\trenchBottom) rectangle (42.25,\STIIslandSurface+2.0);

% substrate
\fill[substrate] (0,0) rectangle (55,\trenchBottom);
\node at (2,0.5) {Silicon substrate};

% normal wells
\fill[substrate] (1.25,\trenchBottom) rectangle (8.25,\STIIslandSurface);
\fill[substrate] (9.75,\trenchBottom) rectangle (16.75,\STIIslandSurface);
\fill[substrate] (18.25,\trenchBottom) rectangle (25.25,\STIIslandSurface);
\fill[substrate] (26.75,\trenchBottom) rectangle (33.75,\STIIslandSurface);
\fill[substrate] (35.25,\trenchBottom) rectangle (42.25,\STIIslandSurface);

	\end{tikzpicture}
	\caption{Trench etching}
\end{figure}

Typically it's a good approach to set the parameters of the DRIE recipe to an amount of cycles which will result in a depth of roughly 5 microns.


\subsection{LTO+CMP}

Now we trenches need to be filled up with LTO and planarized until we meet a sufficiently low height differential between the oxide surface and the silicon surface.

\begin{figure}[H]
	\centering
	\begin{tikzpicture}[node distance = 3cm, auto, thick,scale=\CrossSectionOnly, every node/.style={transform shape}]
		% substrate
\fill[substrate] (0,0) rectangle (55,\trenchBottom);
\node at (2,0.5) {Silicon substrate};

% normal wells
\fill[substrate] (1.25,\trenchBottom) rectangle (8.25,\STIIslandSurface);
\fill[substrate] (9.75,\trenchBottom) rectangle (16.75,\STIIslandSurface);
\fill[substrate] (18.25,\trenchBottom) rectangle (25.25,\STIIslandSurface);
\fill[substrate] (26.75,\trenchBottom) rectangle (33.75,\STIIslandSurface);
\fill[substrate] (35.25,\trenchBottom) rectangle (42.25,\STIIslandSurface);

	\end{tikzpicture}
	\drawStepArrow{}
	\begin{tikzpicture}[node distance = 3cm, auto, thick,scale=\CrossSectionOnly, every node/.style={transform shape}]
		% substrate
\fill[isolationoxide] (0,0) rectangle (55,\STIIslandSurface-1.0);

\newcommand{\leftslope}[1]{
\filldraw[line width=0, isolationoxide] (#1-1.0,\STIIslandSurface-1.0) -- (#1,\STIIslandSurface-1.0) -- (#1,\STIIslandSurface);
}
\newcommand{\rightslope}[1]{
\filldraw[line width=0, isolationoxide] (#1,\STIIslandSurface-1.0) -- (#1+1.0,\STIIslandSurface-1.0) -- (#1,\STIIslandSurface);
}

\leftslope{1.25}
\leftslope{9.75}
\leftslope{18.25}
\leftslope{26.75}
\leftslope{35.25}

\rightslope{8.25}
\rightslope{16.75}
\rightslope{25.25}
\rightslope{33.75}
\rightslope{42.25}

% substrate
\fill[substrate] (0,0) rectangle (55,\trenchBottom);
\node at (2,0.5) {Silicon substrate};

% normal wells
\fill[substrate] (1.25,\trenchBottom) rectangle (8.25,\STIIslandSurface);
\fill[substrate] (9.75,\trenchBottom) rectangle (16.75,\STIIslandSurface);
\fill[substrate] (18.25,\trenchBottom) rectangle (25.25,\STIIslandSurface);
\fill[substrate] (26.75,\trenchBottom) rectangle (33.75,\STIIslandSurface);
\fill[substrate] (35.25,\trenchBottom) rectangle (42.25,\STIIslandSurface);

	\end{tikzpicture}
	\caption{Resist removal}
\end{figure}

This process loop until a height differential below 200nm is being reached turned out to be a good approach. We start counting LTO+CMP steps at 1:
\begin{itemize}
\item Deposit 5 microns LTO ($\frac{5\mu}{2^{step-1}}$)
\item CMP around 5 microns LTO
\item Clean in hot ammonia
\item Put into buffered oxide etch or DI:HF (50:1) for one or two minutes, until silicon surface on the islands is free of oxide
\item Repeat until height differential lower than 200nm
\end{itemize}

\newpage
