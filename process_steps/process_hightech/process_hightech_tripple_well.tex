\section{Tripple Well}\label{tripple_well_chapter}

In order to build BiCMOS we need nested wells for getting the vertical diode structures which form the bi junction transistors.

A vertical isolation, which allows us to have some bulk areas on a higher potential than others, and isolated FETs come along from this tripple well architecture for free.

The cross section of the targeted geometry are shown in \autoref{tripple_well_target}

\begin{figure}[H]
	\centering
	\begin{tikzpicture}[node distance = 3cm, auto, thick,scale=\CrossAndTopSectionBig, every node/.style={transform shape}]
		%silicon oxide
\fill[isolationoxide] (0,0) rectangle (55,\STIIslandSurface);

% substrate
\fill[substrate] (0,0) rectangle (55,\trenchBottom);
\node at (2,0.5) {Silicon substrate};

% normal wells
\fill[substrate] (1.25,\trenchBottom) rectangle (8.25,\STIIslandSurface);
\fill[substrate] (9.75,\trenchBottom) rectangle (16.75,\STIIslandSurface);
\fill[substrate] (18.25,\trenchBottom) rectangle (25.25,\STIIslandSurface);
\fill[substrate] (26.75,\trenchBottom) rectangle (33.75,\STIIslandSurface);
\fill[substrate] (35.25,\trenchBottom) rectangle (42.25,\STIIslandSurface);



\paintnwells{3.0}
\paintpwells{3.0}
\paintpbases{1.5}
\paintnbases{0.5}


	\end{tikzpicture}
	\caption{Tripple well target geometry}
	\label{tripple_well_target}
\end{figure}

Since the diffusion constant variates with the concentration of background dopants, we have to make sure that the thermal budget has enough slack during every single tripple well formation step, in order to avoid the consumption of one of the wells during further processing.

\begin{figure}[H]
	\centering
	\begin{tikzpicture}[node distance =1cm, auto, thick,scale=\VLSILayout, every node/.style={transform shape}]
		\input{tikz_process_steps/pwell.layout.tex}
	\end{tikzpicture}
	\caption{P-Well layout}
	\label{pwell_layout}
\end{figure}

In \autoref{pwell_layout} the layout of the well and base regions on top of the active area region can be seen.

The implant values are as calculated in the documentation of the process design leading to these steps\footnote{\url{https://github.com/leviathanch/libresiliconprocess/raw/master/process_design/process_design.pdf}}.

\newpage
\section{N-well}\label{nwell_chapter}
In order to build CMOS on the same substrate, an N-well is required for building the complementary P-channel transistor for a NFET+PFET logic circuitry.

The cross section as well as the top view of the targeted geometry are shown in \autoref{nwell_target}

\begin{figure}[H]
	\centering
	\begin{tikzpicture}[node distance = 3cm, auto, thick,scale=\CrossAndTopSectionBig, every node/.style={transform shape}]
		%silicon oxide
\fill[isolationoxide] (0,0) rectangle (55,\STIIslandSurface);

% substrate
\fill[substrate] (0,0) rectangle (55,\trenchBottom);
\node at (2,0.5) {Silicon substrate};

% normal wells
\fill[substrate] (1.25,\trenchBottom) rectangle (8.25,\STIIslandSurface);
\fill[substrate] (9.75,\trenchBottom) rectangle (16.75,\STIIslandSurface);
\fill[substrate] (18.25,\trenchBottom) rectangle (25.25,\STIIslandSurface);
\fill[substrate] (26.75,\trenchBottom) rectangle (33.75,\STIIslandSurface);
\fill[substrate] (35.25,\trenchBottom) rectangle (42.25,\STIIslandSurface);



\paintnwells{0.5}

	\end{tikzpicture}
	\begin{tikzpicture}[node distance = 3cm, auto, thick,scale=\CrossAndTopSectionBig, every node/.style={transform shape}]
		% trench area
\fill[isolationoxide] (0,0) rectangle (55,9);

\fill[substrate] (1.25,1) rectangle (8.25,7.25);
\fill[substrate] (9.75,1) rectangle (16.75,7.25);
\fill[substrate] (18.25,1) rectangle (25.25,7.25);
\fill[substrate] (26.75,1) rectangle (33.75,7.25);
\fill[substrate] (35.25,1) rectangle (42.25,7.25);

\fill[nwell] (1.25,1) rectangle (8.25,7.25);
\fill[nwell] (18.25,1) rectangle (25.25,7.25);
\fill[nwell] (26.75,1) rectangle (33.75,7.25);
\fill[nwell] (35.25,1) rectangle (42.25,7.25);
	\end{tikzpicture}
	\caption{N-well target geometry}
	\label{nwell_target}
\end{figure}

The N-well will serve us as an island of N-doped substrate within the P-doped basis substrate.

The P-dopant concentration of our prime grade, p-type, single side polished, four inch wafers is between $8.76 \cdot 10^14 \frac{1}{cm^3}$ and $5.23 \cdot 10^14 \frac{1}{cm^3}$

This means we need a dose of $2.33\times10^{12}cm^{-2}$ Phosphorus at 70 keV, as calculated in the documentation of the process design leading to these steps\footnote{\url{https://github.com/leviathanch/libresiliconprocess/raw/master/process_design/process_design.pdf}}.

The concentration will need adjustment when the used substrate has different properties!

\begin{figure}[H]
	\centering
	\begin{tikzpicture}[node distance =1cm, auto, thick,scale=\VLSILayout, every node/.style={transform shape}]
		\input{tikz_process_steps/nwell.layout.tex}
	\end{tikzpicture}
	\caption{N-Well layout}
	\label{nwell_layout}
\end{figure}

In \autoref{nwell_layout} the layout of the n-well region on top of the active area region can be seen.

The n-well is being fit into the active area. It should even be a little bit bigger than the active area, because of possible alignment offsets.

The layout is being automatically generated for GDS2 based on cifoutput rules, so you just have to draw you well.

\newpage


\subsection{P-well}\label{pwell_chapter}

In order to build CMOS on the same substrate, a P-well is required for building the complementary N-channel transistor for a NFET+PFET logic circuitry.

The cross section as well as the top view of the targeted geometry are shown in \autoref{nwell_target}

\begin{figure}[H]
	\centering
	\begin{tikzpicture}[node distance = 3cm, auto, thick,scale=\CrossAndTopSectionBig, every node/.style={transform shape}]
		%silicon oxide
\fill[isolationoxide] (0,0) rectangle (55,\STIIslandSurface);

% substrate
\fill[substrate] (0,0) rectangle (55,\trenchBottom);
\node at (2,0.5) {Silicon substrate};

% normal wells
\fill[substrate] (1.25,\trenchBottom) rectangle (8.25,\STIIslandSurface);
\fill[substrate] (9.75,\trenchBottom) rectangle (16.75,\STIIslandSurface);
\fill[substrate] (18.25,\trenchBottom) rectangle (25.25,\STIIslandSurface);
\fill[substrate] (26.75,\trenchBottom) rectangle (33.75,\STIIslandSurface);
\fill[substrate] (35.25,\trenchBottom) rectangle (42.25,\STIIslandSurface);



\paintnwells{1.5}
\paintpwells{1.5}

	\end{tikzpicture}
	\begin{tikzpicture}[node distance = 3cm, auto, thick,scale=\CrossAndTopSectionBig, every node/.style={transform shape}]
		\input{tikz_process_steps/sti.b.tex}

\fill[nwell] (1.25,1) rectangle (8.25,7.25);
\fill[nwell] (18.25,1) rectangle (25.25,7.25);
\fill[nwell] (26.75,1) rectangle (33.75,7.25);
\fill[nwell] (35.25,1) rectangle (42.25,7.25);

\fill[pwell] (9.75,1) rectangle (16.75,7.25);
	\end{tikzpicture}
	\caption{P-well target geometry}
	\label{pwell_target}
\end{figure}

The "P-well" will serve us as an island of higher p-doped substrate within the slightly p-doped basis substrate and gives us more flexibility with suppliers, because we can just adjust the doping in case the concentration might be different with another supplier.

The P-dopant concentration of our prime grade, p-type, single side polished, four inch wafers is between $8.76 \cdot 10^{14} \frac{1}{cm^3}$ and $5.23 \cdot 10^{14} \frac{1}{cm^3}$

This means we need a dose of $1.93\times10^{12}cm^{-2}$ Boron atoms at 40 keV.

The concentration will need adjustment when the used substrate has different properties!

After the implantation we perform a drive-in in inert atmosphere at $1050\degreesC$ for two hours and don't have to worry about the substrate anymore.

\newpage
\subsection{P-base}\label{pbase_chapter}

In order to build BiCMOS on the same substrate, a nested P-well within the N-well (now it's twin well) is required for building the bijunction transistors.

The cross section as well as the top view of the targeted geometry are shown in \autoref{pbase_target}

\begin{figure}[H]
	\centering
	\begin{tikzpicture}[node distance = 3cm, auto, thick,scale=\CrossAndTopSectionBig, every node/.style={transform shape}]
		%silicon oxide
\fill[isolationoxide] (0,0) rectangle (55,\STIIslandSurface);

% substrate
\fill[substrate] (0,0) rectangle (55,\trenchBottom);
\node at (2,0.5) {Silicon substrate};

% normal wells
\fill[substrate] (1.25,\trenchBottom) rectangle (8.25,\STIIslandSurface);
\fill[substrate] (9.75,\trenchBottom) rectangle (16.75,\STIIslandSurface);
\fill[substrate] (18.25,\trenchBottom) rectangle (25.25,\STIIslandSurface);
\fill[substrate] (26.75,\trenchBottom) rectangle (33.75,\STIIslandSurface);
\fill[substrate] (35.25,\trenchBottom) rectangle (42.25,\STIIslandSurface);



\paintnwells{2.5}
\paintpwells{2.5}
\paintpbases{1.0}

	\end{tikzpicture}
	\caption{P-base cross section}
	\label{pbase_target}
\end{figure}

The P-base will serve us as an island of higher P-doped substrate within the slightly N-well basis substrate, which will result in a isolated area by forming PN junction versus PN junction.

The dopant dose will be $1.93\times10^{12}cm^{-2}$ at 40 keV.

The P-base can very well cover the N-well area since the expansion mostly is vertical, but it should be kept in mind, that there is also a lateral diffusion when placing contacts also on N-well around the P-base.

After the implantation we perform a drive-in in inert atmosphere at $1050\degreesC$ for one hour.

\section{N-base}\label{nbase_chapter}
In order to build BiCMOS on the same substrate, another N-well within the P-Base (tripple well!) is required for building the complementary isolated P-channel transistor for a n-p-channel logic circuitry as shown above in the example section.

The cross section as well as the top view of the targeted geometry are shown in \autoref{nbase_target}

\begin{figure}[H]
	\centering
	\begin{tikzpicture}[node distance = 3cm, auto, thick,scale=\CrossAndTopSectionBig, every node/.style={transform shape}]
		%silicon oxide
\fill[isolationoxide] (0,0) rectangle (55,\STIIslandSurface);

% substrate
\fill[substrate] (0,0) rectangle (55,\trenchBottom);
\node at (2,0.5) {Silicon substrate};

% normal wells
\fill[substrate] (1.25,\trenchBottom) rectangle (8.25,\STIIslandSurface);
\fill[substrate] (9.75,\trenchBottom) rectangle (16.75,\STIIslandSurface);
\fill[substrate] (18.25,\trenchBottom) rectangle (25.25,\STIIslandSurface);
\fill[substrate] (26.75,\trenchBottom) rectangle (33.75,\STIIslandSurface);
\fill[substrate] (35.25,\trenchBottom) rectangle (42.25,\STIIslandSurface);



\paintnwells{3.0}
\paintpwells{3.0}
\paintpbases{1.5}
\paintnbases{0.5}


	\end{tikzpicture}
	\begin{tikzpicture}[node distance = 3cm, auto, thick,scale=\CrossAndTopSectionBig, every node/.style={transform shape}]
		\input{tikz_process_steps/nbase.b.tex}
	\end{tikzpicture}
	\caption{N-well target geometry}
	\label{nbase_target}
\end{figure}

The N-well will serve us as an island of N-doped substrate within the P-doped basis substrate.

The dopant dose will be $2.33\times10^{12}cm^{-2}$ at 70 keV, as calculated in the documentation of the process design leading to these steps\footnote{\url{https://github.com/leviathanch/libresiliconprocess/raw/master/process_design/process_design.pdf}}.

%\begin{figure}[H]
%	\centering
%	\begin{tikzpicture}[node distance =1cm, auto, thick,scale=\VLSILayout, every node/.style={transform shape}]
%		\input{tikz_process_steps/nbase.layout.tex}
%	\end{tikzpicture}
%	\caption{N-Well layout}
%	\label{nwell_layout}
%\end{figure}

% In \autoref{nbase_layout} the layout of the n-well region on top of the active area region can be seen.

The n-well is being fit into the active area. It should even be a little bit bigger than the active area, because of possible alignment offsets.

The layout is being automatically generated for GDS2 based on cifoutput rules, so you just have to draw you well.

\newpage



