\subsection{N-well}\label{nwell_chapter}
In order to build CMOS on the same substrate, an N-well is required for building the complementary P-channel transistor for a NFET+PFET logic circuitry.

The cross section as well as the top view of the targeted geometry are shown in \autoref{nwell_target}

\begin{figure}[H]
	\centering
	\begin{tikzpicture}[node distance = 3cm, auto, thick,scale=\CrossAndTopSectionBig, every node/.style={transform shape}]
		%silicon oxide
\fill[isolationoxide] (0,0) rectangle (55,\STIIslandSurface);

% substrate
\fill[substrate] (0,0) rectangle (55,\trenchBottom);
\node at (2,0.5) {Silicon substrate};

% normal wells
\fill[substrate] (1.25,\trenchBottom) rectangle (8.25,\STIIslandSurface);
\fill[substrate] (9.75,\trenchBottom) rectangle (16.75,\STIIslandSurface);
\fill[substrate] (18.25,\trenchBottom) rectangle (25.25,\STIIslandSurface);
\fill[substrate] (26.75,\trenchBottom) rectangle (33.75,\STIIslandSurface);
\fill[substrate] (35.25,\trenchBottom) rectangle (42.25,\STIIslandSurface);



\paintnwells{0.5}

	\end{tikzpicture}
	\begin{tikzpicture}[node distance = 3cm, auto, thick,scale=\CrossAndTopSectionBig, every node/.style={transform shape}]
		% trench area
\fill[isolationoxide] (0,0) rectangle (55,9);

\fill[substrate] (1.25,1) rectangle (8.25,7.25);
\fill[substrate] (9.75,1) rectangle (16.75,7.25);
\fill[substrate] (18.25,1) rectangle (25.25,7.25);
\fill[substrate] (26.75,1) rectangle (33.75,7.25);
\fill[substrate] (35.25,1) rectangle (42.25,7.25);

\fill[nwell] (1.25,1) rectangle (8.25,7.25);
\fill[nwell] (18.25,1) rectangle (25.25,7.25);
\fill[nwell] (26.75,1) rectangle (33.75,7.25);
\fill[nwell] (35.25,1) rectangle (42.25,7.25);
	\end{tikzpicture}
	\caption{N-well target geometry}
	\label{nwell_target}
\end{figure}

The N-well will serve us as an island of N-doped substrate within the P-doped basis substrate.

The P-dopant concentration of our prime grade, p-type, single side polished, four inch wafers is between $8.76 \cdot 10^{14} \frac{1}{cm^3}$ and $5.23 \cdot 10^{14} \frac{1}{cm^3}$

This means we need a dose of $2.33\times10^{12}cm^{-2}$ Phosphorus at 70 keV.

The concentration will need adjustment when the used substrate has different properties!
