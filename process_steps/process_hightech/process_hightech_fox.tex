\section{Field oxide}\label{fox_chapter}

The geometry of a substrate with the field oxide filling the shallow trenches from \autoref{sti_chapter} now needs to be made.

\begin{figure}[H]
	\centering
	\begin{tikzpicture}[node distance = 3cm, auto, thick,scale=\CrossAndTopSectionBig, every node/.style={transform shape}]
		%silicon oxide
\fill[isolationoxide] (0,0) rectangle (55,\STIIslandSurface);

% substrate
\fill[substrate] (0,0) rectangle (55,\trenchBottom);
\node at (2,0.5) {Silicon substrate};

% normal wells
\fill[substrate] (1.25,\trenchBottom) rectangle (8.25,\STIIslandSurface);
\fill[substrate] (9.75,\trenchBottom) rectangle (16.75,\STIIslandSurface);
\fill[substrate] (18.25,\trenchBottom) rectangle (25.25,\STIIslandSurface);
\fill[substrate] (26.75,\trenchBottom) rectangle (33.75,\STIIslandSurface);
\fill[substrate] (35.25,\trenchBottom) rectangle (42.25,\STIIslandSurface);



\paintnwells{3.0}
\paintpwells{3.0}
\paintpbases{1.5}
\paintnbases{0.5}



\fill[isolationoxide] (0,\STIIslandSurface) rectangle (1.25,\STIIslandSurface+0.75);
\filldraw[line width=0, isolationoxide] (1.25,\STIIslandSurface+0.75) -- (1.25,\STIIslandSurface) -- (1.35,\STIIslandSurface);

\bjtstopper{2.35}

\stopper{8.15}{1.5}

\bjtstopper{15.25}

\stopper{16.6}{2.4}

\bjtstopper{20.2}

\stopper{24.2}{2.5}

\bjtstopper{27.90}
\bjtstopper{28.00}

\bjtstopper{29.35}
\bjtstopper{29.45}

\bjtstopper{30.80}
\bjtstopper{30.90}

\bjtstopper{32.15}
\bjtstopper{32.25}

\stopper{33.6}{1.80}

\bjtstopper{36.50}
\bjtstopper{36.60}

\bjtstopper{37.85}
\bjtstopper{37.95}

\bjtstopper{39.20}
\bjtstopper{39.30}

\bjtstopper{40.55}
\bjtstopper{40.65}

\filldraw[line width=0, isolationoxide] (41.9,\STIIslandSurface) -- (42.0,\STIIslandSurface) -- (42.0,\STIIslandSurface+0.75);
\fill[isolationoxide] (42.00,\STIIslandSurface) rectangle (55.0,\STIIslandSurface+0.75);


	\end{tikzpicture}
	\caption{Shallow trench isolation target geometry}
	\label{fox_target}
\end{figure}

As can be seen in \autoref{fox_target}, the islands need to be covered with silicon oxide and windows need to be etched into the oxide so that the gate can be constructed later on.

The covering oxide and windows are needed so that the poly silicon is far enough away from the non-active areas so that the threshold voltage of the parasitic FETs is so high that they will never switch.

Only within the active areas we want to allow the poly layer to touch down closer to the silicon.

The mask is called "fox" on the mask set.

The LTO thickness has been chosen to be 200nm which is thin enough for the polysilicon gates to overcome the height difference without damage and still being enough for eliminating parasitic effects.

\newpage

\subsection{Oxide deposition}

Now we need to deposit the silicon dioxide which will provide a spacer between the non active area and the polysilicon gate layer. within the non-active areas.

\begin{figure}[H]
	\centering
	\begin{tikzpicture}[node distance = 3cm, auto, thick,scale=\CrossSectionOnly, every node/.style={transform shape}]
		%silicon oxide
\fill[isolationoxide] (0,0) rectangle (55,\STIIslandSurface);

% substrate
\fill[substrate] (0,0) rectangle (55,\trenchBottom);
\node at (2,0.5) {Silicon substrate};

% normal wells
\fill[substrate] (1.25,\trenchBottom) rectangle (8.25,\STIIslandSurface);
\fill[substrate] (9.75,\trenchBottom) rectangle (16.75,\STIIslandSurface);
\fill[substrate] (18.25,\trenchBottom) rectangle (25.25,\STIIslandSurface);
\fill[substrate] (26.75,\trenchBottom) rectangle (33.75,\STIIslandSurface);
\fill[substrate] (35.25,\trenchBottom) rectangle (42.25,\STIIslandSurface);



\paintnwells{3.0}
\paintpwells{3.0}
\paintpbases{1.5}
\paintnbases{0.5}



	\end{tikzpicture}
	\drawStepArrow{CVD}
	\begin{tikzpicture}[node distance = 3cm, auto, thick,scale=\CrossSectionOnly, every node/.style={transform shape}]
		% oxide
\fill[isolationoxide] (0,\STIIslandSurface) rectangle (55,\STIIslandSurface+0.75);

%silicon oxide
\fill[isolationoxide] (0,0) rectangle (55,\STIIslandSurface);

% substrate
\fill[substrate] (0,0) rectangle (55,\trenchBottom);
\node at (2,0.5) {Silicon substrate};

% normal wells
\fill[substrate] (1.25,\trenchBottom) rectangle (8.25,\STIIslandSurface);
\fill[substrate] (9.75,\trenchBottom) rectangle (16.75,\STIIslandSurface);
\fill[substrate] (18.25,\trenchBottom) rectangle (25.25,\STIIslandSurface);
\fill[substrate] (26.75,\trenchBottom) rectangle (33.75,\STIIslandSurface);
\fill[substrate] (35.25,\trenchBottom) rectangle (42.25,\STIIslandSurface);



\paintnwells{3.0}
\paintpwells{3.0}
\paintpbases{1.5}
\paintnbases{0.5}



	\end{tikzpicture}
	\caption{LTO deposition}
\end{figure}

We deposit a roughly 200nm thick layer of LTO by putting the wafer into the LPCVD furnace.

\subsection{FOX opening formation}\label{fox_etch}

We open the access to the silicon inside of the active areas in order to touch down with the polysilicon further on.

\begin{figure}[H]
	\centering
	\begin{tikzpicture}[node distance = 3cm, auto, thick,scale=\CrossSectionOnly, every node/.style={transform shape}]
		\input{tikz_process_steps/fox.etching.a.tex}
	\end{tikzpicture}
	\drawStepArrow{Dry etch}
	\begin{tikzpicture}[node distance = 3cm, auto, thick,scale=\CrossSectionOnly, every node/.style={transform shape}]
		\input{tikz_process_steps/fox.etching.b.tex}
	\end{tikzpicture}
	\caption{LTO etching}
\end{figure}

The etching time variates from machine to machine and recipe to recipe. Do the math.

After having etched through the LTO we have to make sure that the etching time goes up compared to the undensified LTO+nitride,
which will be etched in \autoref{sonos_chapter}.
For this reason, we put the wafer into the furnace and anneal the LTO for 30 minutes at 850\degreesC in inert atomsphere ($N_2$/$Ar$).

