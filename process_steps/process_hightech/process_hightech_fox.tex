\section{Field oxide}\label{fox_chapter}

The geometry of a substrate with the field oxide filling the shallow trenched from \autoref{sti_chapter} now needs to be made.

\begin{figure}[H]
	\centering
	\begin{tikzpicture}[node distance = 3cm, auto, thick,scale=\CrossAndTopSectionBig, every node/.style={transform shape}]
		%silicon oxide
\fill[isolationoxide] (0,0) rectangle (55,\STIIslandSurface);

% substrate
\fill[substrate] (0,0) rectangle (55,\trenchBottom);
\node at (2,0.5) {Silicon substrate};

% normal wells
\fill[substrate] (1.25,\trenchBottom) rectangle (8.25,\STIIslandSurface);
\fill[substrate] (9.75,\trenchBottom) rectangle (16.75,\STIIslandSurface);
\fill[substrate] (18.25,\trenchBottom) rectangle (25.25,\STIIslandSurface);
\fill[substrate] (26.75,\trenchBottom) rectangle (33.75,\STIIslandSurface);
\fill[substrate] (35.25,\trenchBottom) rectangle (42.25,\STIIslandSurface);



\paintnwells{3.0}
\paintpwells{3.0}
\paintpbases{1.5}
\paintnbases{0.5}



\fill[isolationoxide] (0,\STIIslandSurface) rectangle (1.25,\STIIslandSurface+0.75);
\filldraw[line width=0, isolationoxide] (1.25,\STIIslandSurface+0.75) -- (1.25,\STIIslandSurface) -- (1.35,\STIIslandSurface);

\bjtstopper{2.35}

\stopper{8.15}{1.5}

\bjtstopper{15.25}

\stopper{16.6}{2.4}

\bjtstopper{20.2}

\stopper{24.2}{2.5}

\bjtstopper{27.90}
\bjtstopper{28.00}

\bjtstopper{29.35}
\bjtstopper{29.45}

\bjtstopper{30.80}
\bjtstopper{30.90}

\bjtstopper{32.15}
\bjtstopper{32.25}

\stopper{33.6}{1.80}

\bjtstopper{36.50}
\bjtstopper{36.60}

\bjtstopper{37.85}
\bjtstopper{37.95}

\bjtstopper{39.20}
\bjtstopper{39.30}

\bjtstopper{40.55}
\bjtstopper{40.65}

\filldraw[line width=0, isolationoxide] (41.9,\STIIslandSurface) -- (42.0,\STIIslandSurface) -- (42.0,\STIIslandSurface+0.75);
\fill[isolationoxide] (42.00,\STIIslandSurface) rectangle (55.0,\STIIslandSurface+0.75);


	\end{tikzpicture}
	\begin{tikzpicture}[node distance = 3cm, auto, thick,scale=\CrossAndTopSectionBig, every node/.style={transform shape}]
		\input{tikz_process_steps/fox.b.tex}
	\end{tikzpicture}
	\caption{Shallow trench isolation target geometry}
	\label{fox_target}
\end{figure}

As can be seen in \autoref{fox_target}, the STI trenches need to be filled with silicon oxide and windows need to be etched into them so that the gate can be constructed later on.
The windows are needed so that the poly silicon is far enough away from the non-active areas so that the threshold voltage of the parasitic FETs is so high that they will never switch.
Only within the active areas we want to allow the poly layer to touch down closer to the silicon. \\
During the oxidation the dopants will be further driven in which will lead to the final formation of the N-well with an approximate depth of 4\um and the P-well with an approximate depth of 5\um.

\newpage

\subsection{Oxide deposition}
Now we need to fill the trenches with silicon dioxide which will provide a spacer between the non active area and the polysilicon gate layer. within the non-active areas.

\begin{figure}[H]
	\centering
	\begin{tikzpicture}[node distance = 3cm, auto, thick,scale=\CrossSectionOnly, every node/.style={transform shape}]
		%silicon oxide
\fill[isolationoxide] (0,0) rectangle (55,\STIIslandSurface);

% substrate
\fill[substrate] (0,0) rectangle (55,\trenchBottom);
\node at (2,0.5) {Silicon substrate};

% normal wells
\fill[substrate] (1.25,\trenchBottom) rectangle (8.25,\STIIslandSurface);
\fill[substrate] (9.75,\trenchBottom) rectangle (16.75,\STIIslandSurface);
\fill[substrate] (18.25,\trenchBottom) rectangle (25.25,\STIIslandSurface);
\fill[substrate] (26.75,\trenchBottom) rectangle (33.75,\STIIslandSurface);
\fill[substrate] (35.25,\trenchBottom) rectangle (42.25,\STIIslandSurface);



\paintnwells{3.0}
\paintpwells{3.0}
\paintpbases{1.5}
\paintnbases{0.5}



	\end{tikzpicture}
	\drawStepArrow{}
	\begin{tikzpicture}[node distance = 3cm, auto, thick,scale=\CrossSectionOnly, every node/.style={transform shape}]
		% oxide
\fill[isolationoxide] (0,\STIIslandSurface) rectangle (55,\STIIslandSurface+0.75);

%silicon oxide
\fill[isolationoxide] (0,0) rectangle (55,\STIIslandSurface);

% substrate
\fill[substrate] (0,0) rectangle (55,\trenchBottom);
\node at (2,0.5) {Silicon substrate};

% normal wells
\fill[substrate] (1.25,\trenchBottom) rectangle (8.25,\STIIslandSurface);
\fill[substrate] (9.75,\trenchBottom) rectangle (16.75,\STIIslandSurface);
\fill[substrate] (18.25,\trenchBottom) rectangle (25.25,\STIIslandSurface);
\fill[substrate] (26.75,\trenchBottom) rectangle (33.75,\STIIslandSurface);
\fill[substrate] (35.25,\trenchBottom) rectangle (42.25,\STIIslandSurface);



\paintnwells{3.0}
\paintpwells{3.0}
\paintpbases{1.5}
\paintnbases{0.5}



	\end{tikzpicture}
	\caption{Hard mask growth}
\end{figure}

We grow a roughly 1.23\um thick layer of silicon dioxide by putting the wafer into the furnace at 1050\degreesC for 4 hours and 30 minutes in a wet environment.

During the oxidation the dopants will be further driven in which will lead to the final formation of the N-well with an approximate depth of 4\um and the P-well with an approximate depth of 5\um.

\newpage

\subsection{Etching}\label{fox_etch}

We open the access to the silicon inside of the active areas in order to touch down with the polysilicon further on.

\begin{figure}[H]
	\centering
	\begin{tikzpicture}[node distance = 3cm, auto, thick,scale=\CrossSectionOnly, every node/.style={transform shape}]
		\input{tikz_process_steps/fox.etching.a.tex}
	\end{tikzpicture}
	\drawStepArrow{}
	\begin{tikzpicture}[node distance = 3cm, auto, thick,scale=\CrossSectionOnly, every node/.style={transform shape}]
		\input{tikz_process_steps/fox.etching.b.tex}
	\end{tikzpicture}
	\caption{Nitride mask etching}
\end{figure}

There are dry etching and wet etching methods available for etching the thick field oxide.
The downside of wet etching is that it also etches horizontally, however the chemical BHF is readily available and allows for easy implementation of the process.\\

\textbf{Possible approaches}:
\begin{itemize}
	\item \textbf{"AOE Etcher (DRY-AOE)" from HKUST} \\
	We can use anisotropic plasma etching for sharper borders.
	\item \textbf{Chemical solution} \\
	We can use buffered hydrofluoric acid (BOE (1:6)) at room temperature for a little bit over 3 minutes in order to get through the 300nm of oxide.\\
	Too long over 3 minutes might cause under-etch however!
\end{itemize}

\newpage
