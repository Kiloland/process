\section{Silicification}\label{step_silicification}

Titanium silicide is one of the first SALICIDE material introduced in ULSI devices owing to its low resistivity, high thermal stability, ease in deposition and compatibility with silicon processes.
Titanium has been one of the familiar materials in ULSI productions, which is also an important advantage in practical use of titanium SALICIDE.\footnote{A Study on Formation of High Resistivity Phases of Nickel Silicide at Small Area and its Solution for Scaled CMOS Devices, 07D53437, Ryuji Tomita}

\begin{figure}[H]
	\centering
	\begin{tikzpicture}[node distance = 3cm, auto, thick,scale=\CrossAndTopSectionBig, every node/.style={transform shape}]
		\input{tikz_process_steps/gate.a.tex}
\fill[isolationoxide] (45.0,\polytop+0.75) rectangle (46.0,\implantstoptop);


\paintnimplants{0.5}
\paintpimplants{0.5}



\filldraw[line width=0, nitride] (5.00,\STIIslandSurface) -- (4.50,\STIIslandSurface) -- (5.00,\STIIslandSurface+1.4);
\filldraw[line width=0, nitride] (6.00,\STIIslandSurface) -- (6.50,\STIIslandSurface) -- (6.00,\STIIslandSurface+1.4);

\filldraw[line width=0, nitride] (12.00,\STIIslandSurface) -- (11.50,\STIIslandSurface) -- (12.00,\STIIslandSurface+1.4);
\filldraw[line width=0, nitride] (13.50,\STIIslandSurface) -- (13.00,\STIIslandSurface) -- (13.00,\STIIslandSurface+1.4);

\filldraw[line width=0, nitride] (21.40,\STIIslandSurface) -- (21.90,\STIIslandSurface) -- (21.90,\STIIslandSurface+1.4);
\filldraw[line width=0, nitride] (22.90,\STIIslandSurface) -- (23.40,\STIIslandSurface) -- (22.90,\STIIslandSurface+1.4);


\fill[nitride] (44.0,\polytop+0.75) rectangle (44.8,\polytop+0.95);
\fill[nitride] (44.8,\polytop+0.75) rectangle (45.0,\implantstoptop);
\fill[nitride] (44.8,\implantstoptop) rectangle (46.2,\implantstoptop+0.2);
\fill[nitride] (46.0,\polytop+0.75) rectangle (46.2,\implantstoptop);
\fill[nitride] (46.2,\polytop+0.75) rectangle (47.0,\polytop+0.95);

\fill[silicide] ( 1.35,\STIIslandSurface-0.1) rectangle ( 2.35,\STIIslandSurface);
\fill[silicide] ( 2.70,\STIIslandSurface-0.1) rectangle ( 4.50,\STIIslandSurface);
\fill[silicide] ( 5.00,\STIIslandSurface+1.3) rectangle ( 6.00,\STIIslandSurface+1.4);
\fill[silicide] ( 6.50,\STIIslandSurface-0.1) rectangle ( 8.15,\STIIslandSurface);

\fill[silicide] ( 9.85,\STIIslandSurface-0.1) rectangle (11.50,\STIIslandSurface);
\fill[silicide] (12.00,\STIIslandSurface+1.3) rectangle (13.00,\STIIslandSurface+1.4);
\fill[silicide] (13.50,\STIIslandSurface-0.1) rectangle (15.25,\STIIslandSurface);
\fill[silicide] (15.60,\STIIslandSurface-0.1) rectangle (16.60,\STIIslandSurface);

\fill[silicide] (19.20,\STIIslandSurface-0.1) rectangle (20.20,\STIIslandSurface);
\fill[silicide] (20.60,\STIIslandSurface-0.1) rectangle (21.40,\STIIslandSurface);
\fill[silicide] (21.90,\STIIslandSurface+1.5) rectangle (22.90,\STIIslandSurface+1.6);
\fill[silicide] (23.40,\STIIslandSurface-0.1) rectangle (24.20,\STIIslandSurface);

\fill[silicide] (26.90,\STIIslandSurface-0.1) rectangle (27.90,\STIIslandSurface);
\fill[silicide] (28.35,\STIIslandSurface-0.1) rectangle (29.35,\STIIslandSurface);
\fill[silicide] (29.80,\STIIslandSurface-0.1) rectangle (30.80,\STIIslandSurface);
\fill[silicide] (31.25,\STIIslandSurface-0.1) rectangle (32.15,\STIIslandSurface);
\fill[silicide] (32.60,\STIIslandSurface-0.1) rectangle (33.60,\STIIslandSurface);

\fill[silicide] (35.60,\STIIslandSurface-0.1) rectangle (36.50,\STIIslandSurface);
\fill[silicide] (36.95,\STIIslandSurface-0.1) rectangle (37.85,\STIIslandSurface);
\fill[silicide] (38.30,\STIIslandSurface-0.1) rectangle (39.20,\STIIslandSurface);
\fill[silicide] (39.65,\STIIslandSurface-0.1) rectangle (40.55,\STIIslandSurface);
\fill[silicide] (41.00,\STIIslandSurface-0.1) rectangle (41.90,\STIIslandSurface);

% diode contacts
\fill[silicide] (43.00,\STIIslandSurface+2.0) rectangle (44.00,\STIIslandSurface+2.15);
\fill[silicide] (47.00,\STIIslandSurface+2.0) rectangle (48.00,\STIIslandSurface+2.15);

% resistor contacts
\fill[silicide] (48.50,\STIIslandSurface+2.0) rectangle (49.50,\STIIslandSurface+2.15);
\fill[silicide] (53.50,\STIIslandSurface+2.0) rectangle (54.50,\STIIslandSurface+2.15);

\fill[nitride] (49.5,\polytop+0.75) rectangle (53.5,\polytop+0.95);

	\end{tikzpicture}
	\caption{Silicide geometry target}
	\label{policide_silicide_sections}
\end{figure}

In order to reduce the gate contact resistance as well as the source and drain resistance and in order to provide a more effective etch stop when plasma etching the contact windows to drain, source and gate, silicide/polycide is being added to the wafer as shown in \autoref{policide_silicide_sections}.

The side walls\footnote{\url{http://www.fujitsu.com/jp/group/mifs/en/resources/news/library/tech-intro/process/side-wall.html}} are required in order avoid short circuits between the junction and the gate.

When titanium and silicon are brought into contact and heated at temperatures above 800 \degree C (in the presence of excess silicon) $Ti Si_2$ forms.

The $TiSi_2$ has a resistivity of $12-20 \mu\Omega - cm$.

The basic formation process of titanium SALICIDE is as follows:

A thin titanium film of roughly 30 nm thickness is deposited on an entire wafer with MOSFETs structure.

The deposited Ti film reacts with the exposed silicon areas such as the source/drain area and polysilicon gate electrodes during the annealing at 800\degree C in Argon atmosphere.

Then, the unreacted titanium film on the dielectric layer such as $SiO_2$ or SiN is selectively etched by APM (Ammonia and Hydrogen Peroxide Mixture) solution for around 2-3 minutes.

\newpage

\subsection{Nitride deposition}\label{nitride_spacers_deposition}

The thickness of this CVD deposited nitride layer will be the width of the spacer after having used highly anisotropic etching in the next few steps, for this reason the thickness of the nitride decides over the distance between the silicide and the gate oxide.

Considering, due to the edge effects during dry etching, the thickness of the nitride has to be less than 25\% of the polysilicon thickness, we choose 50nm for the nitride thickness.

\begin{figure}[H]
	\centering
	\begin{tikzpicture}[node distance = 3cm, auto, thick,scale=\CrossSectionOnly, every node/.style={transform shape}]
		\input{tikz_process_steps/gate.a.tex}
\fill[isolationoxide] (45.0,\polytop+0.75) rectangle (46.0,\implantstoptop);


\paintnimplants{0.5}
\paintpimplants{0.5}


	\end{tikzpicture}
	\drawStepArrow{Nitride CVD}
	\begin{tikzpicture}[node distance = 3cm, auto, thick,scale=\CrossSectionOnly, every node/.style={transform shape}]
		\coveringlayer{nitride}{0.5}{0.5}

\filldraw[line width=0, nitride] ( 4.50,\STIIslandSurface) rectangle ( 6.50,\STIIslandSurface+1.9);
\filldraw[line width=0, nitride] (11.50,\STIIslandSurface) rectangle (13.50,\STIIslandSurface+1.9);
\filldraw[line width=0, nitride] (21.40,\STIIslandSurface) rectangle (23.40,\STIIslandSurface+2.1);

\fill[nitride] (42.50,\STIIslandSurface+0.75) rectangle (48.50,\polytop+1.25);
\fill[nitride] (44.5,\polytop+0.75) rectangle (46.5,\implantstoptop+0.5);

\fill[nitride] (48.00,\STIIslandSurface+0.75) rectangle (55.00,\polytop+1.25);

\input{tikz_process_steps/gate.a.tex}
\fill[isolationoxide] (45.0,\polytop+0.75) rectangle (46.0,\implantstoptop);


\paintnimplants{0.5}
\paintpimplants{0.5}



	\end{tikzpicture}
	\caption{Nitride layer}
\end{figure}

The deposition rates might variate between LPCVDs and recipes. It's at the discretion of the operation engineer to achieve those 50nm.

\subsection{Spacer etching}

Now we have to etch our nitride as anisotropic as possible.

This means that the etching mostly only comes "from above" with a few to nearly none horizontal etching.

Thit means the etching process only "sees" the sidewall as a "thicker layer" and starts etching downward.

\begin{figure}[H]
	\centering
	\begin{tikzpicture}[node distance = 3cm, auto, thick,scale=\CrossSectionOnly, every node/.style={transform shape}]
		\coveringlayer{nitride}{0.5}{0.5}

\filldraw[line width=0, nitride] ( 4.50,\STIIslandSurface) rectangle ( 6.50,\STIIslandSurface+1.9);
\filldraw[line width=0, nitride] (11.50,\STIIslandSurface) rectangle (13.50,\STIIslandSurface+1.9);
\filldraw[line width=0, nitride] (21.40,\STIIslandSurface) rectangle (23.40,\STIIslandSurface+2.1);

\fill[nitride] (42.50,\STIIslandSurface+0.75) rectangle (48.50,\polytop+1.25);
\fill[nitride] (44.5,\polytop+0.75) rectangle (46.5,\implantstoptop+0.5);

\fill[nitride] (48.00,\STIIslandSurface+0.75) rectangle (55.00,\polytop+1.25);

\input{tikz_process_steps/implantstop.a.tex}

\paintnimplants{0.5}
\paintpimplants{0.5}




	\end{tikzpicture}
	\drawStepArrow{Dry etching}
	\begin{tikzpicture}[node distance = 3cm, auto, thick,scale=\CrossSectionOnly, every node/.style={transform shape}]
		\filldraw[line width=0, nitride] (5.00,\STIIslandSurface) -- (4.50,\STIIslandSurface) -- (5.00,\STIIslandSurface+1.4);
\filldraw[line width=0, nitride] (6.00,\STIIslandSurface) -- (6.50,\STIIslandSurface) -- (6.00,\STIIslandSurface+1.4);

\filldraw[line width=0, nitride] (12.00,\STIIslandSurface) -- (11.50,\STIIslandSurface) -- (12.00,\STIIslandSurface+1.4);
\filldraw[line width=0, nitride] (13.50,\STIIslandSurface) -- (13.00,\STIIslandSurface) -- (13.00,\STIIslandSurface+1.4);

\filldraw[line width=0, nitride] (21.40,\STIIslandSurface) -- (21.90,\STIIslandSurface) -- (21.90,\STIIslandSurface+1.6);
\filldraw[line width=0, nitride] (22.90,\STIIslandSurface) -- (23.40,\STIIslandSurface) -- (22.90,\STIIslandSurface+1.6);

\fill[nitride] (44.0,\polytop+0.75) rectangle (44.8,\polytop+1.25);
\fill[nitride] (44.5,\polytop+0.75) rectangle (46.5,\implantstoptop+0.5);
\fill[nitride] (46.2,\polytop+0.75) rectangle (47.0,\polytop+1.25);

\fill[nitride] (49.5,\polytop+0.75) rectangle (53.5,\polytop+1.25);

\input{tikz_process_steps/gate.a.tex}
\fill[isolationoxide] (45.0,\polytop+0.75) rectangle (46.0,\implantstoptop);


\paintnimplants{0.5}
\paintpimplants{0.5}




	\end{tikzpicture}
	\caption{Anisotropic etching}
\end{figure}

After that we will have our desired spacer geometry forming as well as any potentially resist covered area from the silicide block mask patterns, which will allow us to control in which areas we reduce the sheet resistance in which we don't.

\newpage

\subsection{Titanium deposition}

We deposit a layer of titanium with a thickness of around 30nm which will then be reacted into titanium-silicide and titanium-polycide respectively in the further steps.

\begin{figure}[H]
	\centering
	\begin{tikzpicture}[node distance = 3cm, auto, thick,scale=\CrossSectionOnly, every node/.style={transform shape}]
		\filldraw[line width=0, nitride] (5.00,\STIIslandSurface) -- (4.50,\STIIslandSurface) -- (5.00,\STIIslandSurface+1.4);
\filldraw[line width=0, nitride] (6.00,\STIIslandSurface) -- (6.50,\STIIslandSurface) -- (6.00,\STIIslandSurface+1.4);

\filldraw[line width=0, nitride] (12.00,\STIIslandSurface) -- (11.50,\STIIslandSurface) -- (12.00,\STIIslandSurface+1.4);
\filldraw[line width=0, nitride] (13.50,\STIIslandSurface) -- (13.00,\STIIslandSurface) -- (13.00,\STIIslandSurface+1.4);

\filldraw[line width=0, nitride] (21.40,\STIIslandSurface) -- (21.90,\STIIslandSurface) -- (21.90,\STIIslandSurface+1.6);
\filldraw[line width=0, nitride] (22.90,\STIIslandSurface) -- (23.40,\STIIslandSurface) -- (22.90,\STIIslandSurface+1.6);

\fill[nitride] (44.0,\polytop+0.75) rectangle (44.8,\polytop+1.25);
\fill[nitride] (44.5,\polytop+0.75) rectangle (46.5,\implantstoptop+0.5);
\fill[nitride] (46.2,\polytop+0.75) rectangle (47.0,\polytop+1.25);

\fill[nitride] (49.5,\polytop+0.75) rectangle (53.5,\polytop+1.25);

\input{tikz_process_steps/implantstop.a.tex}

\paintnimplants{0.5}
\paintpimplants{0.5}





	\end{tikzpicture}
	\drawStepArrow{Sputtering}
	\begin{tikzpicture}[node distance = 3cm, auto, thick,scale=\CrossSectionOnly, every node/.style={transform shape}]
		\paintactivecover{titanium}{0.5}
\filldraw[line width=0, nitride] (5.00,\STIIslandSurface) -- (4.50,\STIIslandSurface) -- (5.00,\STIIslandSurface+1.4);
\filldraw[line width=0, nitride] (6.00,\STIIslandSurface) -- (6.50,\STIIslandSurface) -- (6.00,\STIIslandSurface+1.4);

\filldraw[line width=0, nitride] (12.00,\STIIslandSurface) -- (11.50,\STIIslandSurface) -- (12.00,\STIIslandSurface+1.4);
\filldraw[line width=0, nitride] (13.50,\STIIslandSurface) -- (13.00,\STIIslandSurface) -- (13.00,\STIIslandSurface+1.4);

\filldraw[line width=0, nitride] (21.40,\STIIslandSurface) -- (21.90,\STIIslandSurface) -- (21.90,\STIIslandSurface+1.6);
\filldraw[line width=0, nitride] (22.90,\STIIslandSurface) -- (23.40,\STIIslandSurface) -- (22.90,\STIIslandSurface+1.6);

\fill[nitride] (44.0,\polytop+0.75) rectangle (44.8,\polytop+1.25);
\fill[nitride] (44.5,\polytop+0.75) rectangle (46.5,\implantstoptop+0.5);
\fill[nitride] (46.2,\polytop+0.75) rectangle (47.0,\polytop+1.25);

\fill[nitride] (49.5,\polytop+0.75) rectangle (53.5,\polytop+1.25);

\input{tikz_process_steps/implantstop.a.tex}

\paintnimplants{0.5}
\paintpimplants{0.5}





	\end{tikzpicture}
	\caption{Titanium deposition}
\end{figure}

The titanium can either be applied by sputtering or by chemical deposition.

\subsection{Silicide formation}

The deposited Ti film reacts with the exposed silicon areas such as the source/drain area and polysilicon gate electrodes during RTP (Rapid Thermal Processing) at 800\degreesC in Argon ambient for 30 seconds.

In this annealing step the $Ti Si_2$ is formed.

\begin{figure}[H]
	\centering
	\begin{tikzpicture}[node distance = 3cm, auto, thick,scale=\CrossSectionOnly, every node/.style={transform shape}]
		\paintactivecover{titanium}{0.5}
\filldraw[line width=0, nitride] (5.00,\STIIslandSurface) -- (4.50,\STIIslandSurface) -- (5.00,\STIIslandSurface+1.4);
\filldraw[line width=0, nitride] (6.00,\STIIslandSurface) -- (6.50,\STIIslandSurface) -- (6.00,\STIIslandSurface+1.4);

\filldraw[line width=0, nitride] (12.00,\STIIslandSurface) -- (11.50,\STIIslandSurface) -- (12.00,\STIIslandSurface+1.4);
\filldraw[line width=0, nitride] (13.50,\STIIslandSurface) -- (13.00,\STIIslandSurface) -- (13.00,\STIIslandSurface+1.4);

\filldraw[line width=0, nitride] (21.40,\STIIslandSurface) -- (21.90,\STIIslandSurface) -- (21.90,\STIIslandSurface+1.6);
\filldraw[line width=0, nitride] (22.90,\STIIslandSurface) -- (23.40,\STIIslandSurface) -- (22.90,\STIIslandSurface+1.6);

\fill[nitride] (44.0,\polytop+0.75) rectangle (44.8,\polytop+1.25);
\fill[nitride] (44.5,\polytop+0.75) rectangle (46.5,\implantstoptop+0.5);
\fill[nitride] (46.2,\polytop+0.75) rectangle (47.0,\polytop+1.25);

\fill[nitride] (49.5,\polytop+0.75) rectangle (53.5,\polytop+1.25);

\input{tikz_process_steps/pimplant.a.tex}




	\end{tikzpicture}
	\drawStepArrow{RTP}
	\begin{tikzpicture}[node distance = 3cm, auto, thick,scale=\CrossSectionOnly, every node/.style={transform shape}]
		\paintactivecover{titanium}{0.5}
\filldraw[line width=0, nitride] (5.00,\STIIslandSurface) -- (4.50,\STIIslandSurface) -- (5.00,\STIIslandSurface+1.4);
\filldraw[line width=0, nitride] (6.00,\STIIslandSurface) -- (6.50,\STIIslandSurface) -- (6.00,\STIIslandSurface+1.4);

\filldraw[line width=0, nitride] (12.00,\STIIslandSurface) -- (11.50,\STIIslandSurface) -- (12.00,\STIIslandSurface+1.4);
\filldraw[line width=0, nitride] (13.50,\STIIslandSurface) -- (13.00,\STIIslandSurface) -- (13.00,\STIIslandSurface+1.4);

\filldraw[line width=0, nitride] (21.40,\STIIslandSurface) -- (21.90,\STIIslandSurface) -- (21.90,\STIIslandSurface+1.6);
\filldraw[line width=0, nitride] (22.90,\STIIslandSurface) -- (23.40,\STIIslandSurface) -- (22.90,\STIIslandSurface+1.6);

\fill[nitride] (44.0,\polytop+0.75) rectangle (44.8,\polytop+1.25);
\fill[nitride] (44.5,\polytop+0.75) rectangle (46.5,\implantstoptop+0.5);
\fill[nitride] (46.2,\polytop+0.75) rectangle (47.0,\polytop+1.25);

\fill[nitride] (49.5,\polytop+0.75) rectangle (53.5,\polytop+1.25);

\input{tikz_process_steps/pimplant.a.tex}




\fill[silicide] ( 1.35,\STIIslandSurface-0.1) rectangle ( 2.35,\STIIslandSurface);
\fill[silicide] ( 2.70,\STIIslandSurface-0.1) rectangle ( 4.50,\STIIslandSurface);
\fill[silicide] ( 5.00,\STIIslandSurface+1.3) rectangle ( 6.00,\STIIslandSurface+1.4);
\fill[silicide] ( 6.50,\STIIslandSurface-0.1) rectangle ( 8.15,\STIIslandSurface);

\fill[silicide] ( 9.85,\STIIslandSurface-0.1) rectangle (11.50,\STIIslandSurface);
\fill[silicide] (12.00,\STIIslandSurface+1.3) rectangle (13.00,\STIIslandSurface+1.4);
\fill[silicide] (13.50,\STIIslandSurface-0.1) rectangle (15.25,\STIIslandSurface);
\fill[silicide] (15.60,\STIIslandSurface-0.1) rectangle (16.60,\STIIslandSurface);

\fill[silicide] (19.20,\STIIslandSurface-0.1) rectangle (20.20,\STIIslandSurface);
\fill[silicide] (20.60,\STIIslandSurface-0.1) rectangle (21.40,\STIIslandSurface);
\fill[silicide] (21.90,\STIIslandSurface+1.5) rectangle (22.90,\STIIslandSurface+1.6);
\fill[silicide] (23.40,\STIIslandSurface-0.1) rectangle (24.20,\STIIslandSurface);

\fill[silicide] (26.90,\STIIslandSurface-0.1) rectangle (27.90,\STIIslandSurface);
\fill[silicide] (28.35,\STIIslandSurface-0.1) rectangle (29.35,\STIIslandSurface);
\fill[silicide] (29.80,\STIIslandSurface-0.1) rectangle (30.80,\STIIslandSurface);
\fill[silicide] (31.25,\STIIslandSurface-0.1) rectangle (32.15,\STIIslandSurface);
\fill[silicide] (32.60,\STIIslandSurface-0.1) rectangle (33.60,\STIIslandSurface);

\fill[silicide] (35.60,\STIIslandSurface-0.1) rectangle (36.50,\STIIslandSurface);
\fill[silicide] (36.95,\STIIslandSurface-0.1) rectangle (37.85,\STIIslandSurface);
\fill[silicide] (38.30,\STIIslandSurface-0.1) rectangle (39.20,\STIIslandSurface);
\fill[silicide] (39.65,\STIIslandSurface-0.1) rectangle (40.55,\STIIslandSurface);
\fill[silicide] (41.00,\STIIslandSurface-0.1) rectangle (41.90,\STIIslandSurface);

% diode contacts
\fill[silicide] (43.00,\STIIslandSurface+2.0) rectangle (44.00,\STIIslandSurface+2.15);
\fill[silicide] (47.00,\STIIslandSurface+2.0) rectangle (48.00,\STIIslandSurface+2.15);

% resistor contacts
\fill[silicide] (48.50,\STIIslandSurface+2.0) rectangle (49.50,\STIIslandSurface+2.15);
\fill[silicide] (53.50,\STIIslandSurface+2.0) rectangle (54.50,\STIIslandSurface+2.15);


	\end{tikzpicture}
	\caption{RTP treatment}
\end{figure}

The resulting $Ti Si_2$ film will be around 77nm in tickness with around 20nm unreacted titanium left on top.

A color change into a slightly brownish color from originally silver metallic can be observed of the titanium on top of the oxide.

\newpage

\subsection{Metal removal}

The unreacted titanium film on the dielectric layer such as $SiO_2$ or $SiN$ is selectively etched by APM (Ammonia and Hydrogen Peroxide Mixture) solution.

\begin{figure}[H]
	\centering
	\begin{tikzpicture}[node distance = 3cm, auto, thick,scale=\CrossSectionOnly, every node/.style={transform shape}]
		\paintactivecover{titanium}{0.5}
\input{tikz_process_steps/silicification.sputter_etching.b.tex}


\fill[silicide] ( 1.35,\STIIslandSurface-0.1) rectangle ( 2.35,\STIIslandSurface);
\fill[silicide] ( 2.70,\STIIslandSurface-0.1) rectangle ( 4.50,\STIIslandSurface);
\fill[silicide] ( 5.00,\STIIslandSurface+1.3) rectangle ( 6.00,\STIIslandSurface+1.4);
\fill[silicide] ( 6.50,\STIIslandSurface-0.1) rectangle ( 8.15,\STIIslandSurface);

\fill[silicide] ( 9.85,\STIIslandSurface-0.1) rectangle (11.50,\STIIslandSurface);
\fill[silicide] (12.00,\STIIslandSurface+1.3) rectangle (13.00,\STIIslandSurface+1.4);
\fill[silicide] (13.50,\STIIslandSurface-0.1) rectangle (15.25,\STIIslandSurface);
\fill[silicide] (15.60,\STIIslandSurface-0.1) rectangle (16.60,\STIIslandSurface);

\fill[silicide] (19.20,\STIIslandSurface-0.1) rectangle (20.20,\STIIslandSurface);
\fill[silicide] (20.60,\STIIslandSurface-0.1) rectangle (21.40,\STIIslandSurface);
\fill[silicide] (21.90,\STIIslandSurface+1.5) rectangle (22.90,\STIIslandSurface+1.6);
\fill[silicide] (23.40,\STIIslandSurface-0.1) rectangle (24.20,\STIIslandSurface);

\fill[silicide] (26.90,\STIIslandSurface-0.1) rectangle (27.90,\STIIslandSurface);
\fill[silicide] (28.35,\STIIslandSurface-0.1) rectangle (29.35,\STIIslandSurface);
\fill[silicide] (29.80,\STIIslandSurface-0.1) rectangle (30.80,\STIIslandSurface);
\fill[silicide] (31.25,\STIIslandSurface-0.1) rectangle (32.15,\STIIslandSurface);
\fill[silicide] (32.60,\STIIslandSurface-0.1) rectangle (33.60,\STIIslandSurface);

\fill[silicide] (35.60,\STIIslandSurface-0.1) rectangle (36.50,\STIIslandSurface);
\fill[silicide] (36.95,\STIIslandSurface-0.1) rectangle (37.85,\STIIslandSurface);
\fill[silicide] (38.30,\STIIslandSurface-0.1) rectangle (39.20,\STIIslandSurface);
\fill[silicide] (39.65,\STIIslandSurface-0.1) rectangle (40.55,\STIIslandSurface);
\fill[silicide] (41.00,\STIIslandSurface-0.1) rectangle (41.90,\STIIslandSurface);

% diode contacts
\fill[silicide] (43.00,\STIIslandSurface+2.0) rectangle (44.00,\STIIslandSurface+2.15);
\fill[silicide] (47.00,\STIIslandSurface+2.0) rectangle (48.00,\STIIslandSurface+2.15);

% resistor contacts
\fill[silicide] (48.50,\STIIslandSurface+2.0) rectangle (49.50,\STIIslandSurface+2.15);
\fill[silicide] (53.50,\STIIslandSurface+2.0) rectangle (54.50,\STIIslandSurface+2.15);



	\end{tikzpicture}
	\drawStepArrow{RCA cleaning}
	\begin{tikzpicture}[node distance = 3cm, auto, thick,scale=\CrossSectionOnly, every node/.style={transform shape}]
		\input{tikz_process_steps/implantstop.a.tex}

\paintnimplants{0.5}
\paintpimplants{0.5}



\filldraw[line width=0, nitride] (5.00,\STIIslandSurface) -- (4.50,\STIIslandSurface) -- (5.00,\STIIslandSurface+1.4);
\filldraw[line width=0, nitride] (6.00,\STIIslandSurface) -- (6.50,\STIIslandSurface) -- (6.00,\STIIslandSurface+1.4);

\filldraw[line width=0, nitride] (12.00,\STIIslandSurface) -- (11.50,\STIIslandSurface) -- (12.00,\STIIslandSurface+1.4);
\filldraw[line width=0, nitride] (13.50,\STIIslandSurface) -- (13.00,\STIIslandSurface) -- (13.00,\STIIslandSurface+1.4);

\filldraw[line width=0, nitride] (21.40,\STIIslandSurface) -- (21.90,\STIIslandSurface) -- (21.90,\STIIslandSurface+1.4);
\filldraw[line width=0, nitride] (22.90,\STIIslandSurface) -- (23.40,\STIIslandSurface) -- (22.90,\STIIslandSurface+1.4);


\fill[nitride] (44.0,\polytop+0.75) rectangle (44.8,\polytop+0.95);
\fill[nitride] (44.8,\polytop+0.75) rectangle (45.0,\implantstoptop);
\fill[nitride] (44.8,\implantstoptop) rectangle (46.2,\implantstoptop+0.2);
\fill[nitride] (46.0,\polytop+0.75) rectangle (46.2,\implantstoptop);
\fill[nitride] (46.2,\polytop+0.75) rectangle (47.0,\polytop+0.95);

\fill[silicide] ( 1.35,\STIIslandSurface-0.1) rectangle ( 2.35,\STIIslandSurface);
\fill[silicide] ( 2.70,\STIIslandSurface-0.1) rectangle ( 4.50,\STIIslandSurface);
\fill[silicide] ( 5.00,\STIIslandSurface+1.3) rectangle ( 6.00,\STIIslandSurface+1.4);
\fill[silicide] ( 6.50,\STIIslandSurface-0.1) rectangle ( 8.15,\STIIslandSurface);

\fill[silicide] ( 9.85,\STIIslandSurface-0.1) rectangle (11.50,\STIIslandSurface);
\fill[silicide] (12.00,\STIIslandSurface+1.3) rectangle (13.00,\STIIslandSurface+1.4);
\fill[silicide] (13.50,\STIIslandSurface-0.1) rectangle (15.25,\STIIslandSurface);
\fill[silicide] (15.60,\STIIslandSurface-0.1) rectangle (16.60,\STIIslandSurface);

\fill[silicide] (19.20,\STIIslandSurface-0.1) rectangle (20.20,\STIIslandSurface);
\fill[silicide] (20.60,\STIIslandSurface-0.1) rectangle (21.40,\STIIslandSurface);
\fill[silicide] (21.90,\STIIslandSurface+1.5) rectangle (22.90,\STIIslandSurface+1.6);
\fill[silicide] (23.40,\STIIslandSurface-0.1) rectangle (24.20,\STIIslandSurface);

\fill[silicide] (26.90,\STIIslandSurface-0.1) rectangle (27.90,\STIIslandSurface);
\fill[silicide] (28.35,\STIIslandSurface-0.1) rectangle (29.35,\STIIslandSurface);
\fill[silicide] (29.80,\STIIslandSurface-0.1) rectangle (30.80,\STIIslandSurface);
\fill[silicide] (31.25,\STIIslandSurface-0.1) rectangle (32.15,\STIIslandSurface);
\fill[silicide] (32.60,\STIIslandSurface-0.1) rectangle (33.60,\STIIslandSurface);

\fill[silicide] (35.60,\STIIslandSurface-0.1) rectangle (36.50,\STIIslandSurface);
\fill[silicide] (36.95,\STIIslandSurface-0.1) rectangle (37.85,\STIIslandSurface);
\fill[silicide] (38.30,\STIIslandSurface-0.1) rectangle (39.20,\STIIslandSurface);
\fill[silicide] (39.65,\STIIslandSurface-0.1) rectangle (40.55,\STIIslandSurface);
\fill[silicide] (41.00,\STIIslandSurface-0.1) rectangle (41.90,\STIIslandSurface);

% diode contacts
\fill[silicide] (43.00,\STIIslandSurface+2.0) rectangle (44.00,\STIIslandSurface+2.15);
\fill[silicide] (47.00,\STIIslandSurface+2.0) rectangle (48.00,\STIIslandSurface+2.15);

% resistor contacts
\fill[silicide] (48.50,\STIIslandSurface+2.0) rectangle (49.50,\STIIslandSurface+2.15);
\fill[silicide] (53.50,\STIIslandSurface+2.0) rectangle (54.50,\STIIslandSurface+2.15);

\fill[nitride] (49.5,\polytop+0.75) rectangle (53.5,\polytop+0.95);


	\end{tikzpicture}
	\caption{Titanium etch}
\end{figure}

After 2-3 minutes in APM, at room temperature, with a bit mechanical help, all the unreacted Titanium should be gone and the oxide should become visible again.
Under \textbf{no circumstance} use a solvent containing HF, since $Ti Si_2$ dissolves in HF or any other Fluoride containing solutions.

\subsection{CMP}\label{chapter_silicide_and_cmp}

After we formed all the active devices and added the silicide in order to reduce the sheet resistance of junctions and contacts we have to make sure that
our devices will not be damaged during the planarization phase in order to contact through to them with the first metal layer.

\begin{figure}[H]
	\centering
	\begin{tikzpicture}[node distance = 3cm, auto, thick,scale=\CrossSectionOnly, every node/.style={transform shape}]
		\input{tikz_process_steps/silicification.cmp_stop.a.tex}
	\end{tikzpicture}
	\drawStepArrow{LTO+CMP}
	\begin{tikzpicture}[node distance = 3cm, auto, thick,scale=\CrossSectionOnly, every node/.style={transform shape}]
		\input{tikz_process_steps/silicification.cmp_stop.b.tex}
	\end{tikzpicture}
	\caption{CMP, contact preparation}
\end{figure}

First we deposit around 100nm LTO as a pad oxide layer below the 100nm nitride, which serves as the CMP stop hard mask.
Then we deposit 1\um LTO and CMP away the height differential of the active devices translated to the oxide.

LTO was chosen because the silicide becomes unstable in the thermal ranges where phosphorus silicate glass becomes viscous enough
for evening out the height differential by evening out by seeking its level during annealing.

A thickness of 1\um of the LTO will make it less likely, that the dishing effect of the CMP pad causes devices to get damaged
through an over consumption of the nitride hard mask.

The best approach for depositing this LTO layer is to split the deposition into 4 steps at each 250nm and rotating the sample
90 degrees between the steps in order to improve uniformity of the LTO layer.
