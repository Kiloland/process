\subsection{P-base}\label{pbase_chapter}

In order to build BiCMOS on the same substrate, a nested P-well within the N-well (now it's twin well) is required for building the bijunction transistors.

The cross section as well as the top view of the targeted geometry are shown in \autoref{pbase_target}

\begin{figure}[H]
	\centering
	\begin{tikzpicture}[node distance = 3cm, auto, thick,scale=\CrossAndTopSectionBig, every node/.style={transform shape}]
		%silicon oxide
\fill[isolationoxide] (0,0) rectangle (55,\STIIslandSurface);

% substrate
\fill[substrate] (0,0) rectangle (55,\trenchBottom);
\node at (2,0.5) {Silicon substrate};

% normal wells
\fill[substrate] (1.25,\trenchBottom) rectangle (8.25,\STIIslandSurface);
\fill[substrate] (9.75,\trenchBottom) rectangle (16.75,\STIIslandSurface);
\fill[substrate] (18.25,\trenchBottom) rectangle (25.25,\STIIslandSurface);
\fill[substrate] (26.75,\trenchBottom) rectangle (33.75,\STIIslandSurface);
\fill[substrate] (35.25,\trenchBottom) rectangle (42.25,\STIIslandSurface);



\paintnwells{2.5}
\paintpwells{2.5}
\paintpbases{1.0}

	\end{tikzpicture}
	\caption{P-base cross section}
	\label{pbase_target}
\end{figure}

The P-base will serve us as an island of higher P-doped substrate within the slightly N-well basis substrate, which will result in a isolated area by forming PN junction versus PN junction.

The dopant dose will be $1.93\times10^{12}cm^{-2}$ at 40 keV.

The P-base can very well cover the N-well area since the expansion mostly is vertical, but it should be kept in mind, that there is also a lateral diffusion when placing contacts also on N-well around the P-base.

After the implantation we perform a drive-in in inert atmosphere at $1050\degreesC$ for one hour.
