`\section{P-base}\label{pbase_chapter}
In order to build CMOS on the same substrate, a P-well is required for building the complementary N-channel transistor for a n-p-channel logic circuitry.
The cross section as well as the top view of the targeted geometry are shown in \autoref{nwell_target}
\begin{figure}[H]
	\centering
	\begin{tikzpicture}[node distance = 3cm, auto, thick,scale=\CrossAndTopSectionBig, every node/.style={transform shape}]
		%silicon oxide
\fill[isolationoxide] (0,0) rectangle (55,\STIIslandSurface);

% substrate
\fill[substrate] (0,0) rectangle (55,\trenchBottom);
\node at (2,0.5) {Silicon substrate};

% normal wells
\fill[substrate] (1.25,\trenchBottom) rectangle (8.25,\STIIslandSurface);
\fill[substrate] (9.75,\trenchBottom) rectangle (16.75,\STIIslandSurface);
\fill[substrate] (18.25,\trenchBottom) rectangle (25.25,\STIIslandSurface);
\fill[substrate] (26.75,\trenchBottom) rectangle (33.75,\STIIslandSurface);
\fill[substrate] (35.25,\trenchBottom) rectangle (42.25,\STIIslandSurface);



\paintnwells{2.5}
\paintpwells{2.5}
\paintpbases{1.0}

	\end{tikzpicture}
	\begin{tikzpicture}[node distance = 3cm, auto, thick,scale=\CrossAndTopSectionBig, every node/.style={transform shape}]
		\input{tikz_process_steps/sti.b.tex}

\fill[nwell] (1.25,1) rectangle (8.25,7.25);
\fill[nwell] (18.25,1) rectangle (25.25,7.25);
\fill[nwell] (26.75,1) rectangle (33.75,7.25);
\fill[nwell] (35.25,1) rectangle (42.25,7.25);

\fill[pwell] (9.75,1) rectangle (16.75,7.25);

\fill[pbase] (18.5,1.5) rectangle (25.0,6.75);
\fill[pbase] (28.25,1.5) rectangle (36.75,6.75);
\fill[pbase] (35.5,1.5) rectangle (37.0,6.75);
\fill[pbase] (38.0,1.5) rectangle (39.5,6.75);
\fill[pbase] (40.5,1.5) rectangle (41.5,6.75);


	\end{tikzpicture}
	\caption{P-well target geometry}
	\label{pwell_target}
\end{figure}
The P-well will serve us as an island of higher p-doped substrate within the slightly p-doped basis substrate.

The dopant dose will be $1.93\times10^{12}cm^{-2}$ at 40 keV, as calculated in the documentation of the process design leading to these steps\footnote{\url{https://github.com/leviathanch/libresiliconprocess/raw/master/process_design/process_design.pdf}}.

\begin{figure}[H]
	\centering
	\begin{tikzpicture}[node distance =1cm, auto, thick,scale=\VLSILayout, every node/.style={transform shape}]
		\input{tikz_process_steps/pwell.layout.tex}
	\end{tikzpicture}
	\caption{P-Well layout}
	\label{pwell_layout}
\end{figure}

In \autoref{pwell_layout} the layout of the P-well region on top of the active area region can be seen.

The p-well is being fit into the active area.

It should even be a little bit bigger than the active area, because of possible alignment offsets.

The layout is being automatically generated for GDS2 based on cifoutput rules, so you just have to draw you well.

\newpage

