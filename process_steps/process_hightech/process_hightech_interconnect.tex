\section{Interconnect}

Now that we've built all the devices, we've got to put wires on them in order to make them do something useful.

\begin{figure}[H]
	\centering
	\begin{tikzpicture}[node distance = 3cm, auto, thick,scale=\CrossAndTopSectionBig, every node/.style={transform shape}]
		\fill[isolationoxide] (0.0,\LowerMoreMetalTwo) rectangle (55.0,\UpperGlass);

\fill[isolationoxide] (0.0,\LowerMoreMetal) rectangle (55.0,\LowerMoreMetalTwo);

\input{tikz_process_steps/metal2.a.tex}

\paintscaledvias{white}{\UpperMoreMetal}{\LowerMoreMetalTwo}{0}


\fill[nitride] (0,\LowerMoreMetalTwo) rectangle (55,\LowerMoreMetalTwo+0.5);
\fill[isolationoxide] (0,\LowerMoreMetalTwo) rectangle (55,\LowerMoreMetalTwo+0.25);

\paintscaledvias{nitride}{\LowerMoreMetalTwo+0.25}{\UpperMoreMetalTwo+0.5}{1.00}
\paintscaledvias{isolationoxide}{\LowerMoreMetalTwo}{\UpperMoreMetalTwo+0.25}{0.75}

\paintscaledvias{gray}{\UpperMoreMetal}{\LowerMoreMetalTwo}{0.25}
\paintscaledvias{gray}{\LowerMoreMetalTwo}{\UpperMoreMetalTwo}{0.50}

\paintscaledvias{brown}{\UpperMoreMetal+0.25}{\LowerMoreMetalTwo+0.3}{-0.1}
\paintscaledvias{brown}{\LowerMoreMetalTwo+0.3}{\UpperMoreMetalTwo}{0.5}


\paintscaledvias{white}{\UpperMoreMetalTwo}{\UpperGlass}{0.25}

	\end{tikzpicture}
	\caption{Interconnect geometry target}
	\label{inteconnect_cross_section}
\end{figure}

From here on it's basically just always the same game:
\begin{enumerate}
	\item Deposit roughly 100nm LTO
	\item Deposit 100nm nitride for CMP end stop
	\item Deposit roughly 1\um LTO
	\item CMP away 200nm oxide (regulated by CMP end stop)
	\item Etch vias
	\item Sputter metal
	\item Etch wires
	\item Go to 1
\end{enumerate}

As can be seen in \autoref{inteconnect_cross_section}, we have the holes where we sputter the metal into.

All the oxide holes which are \autoref{chapter_contact}, \autoref{chapter_via1}, \autoref{chapter_via2} and \autoref{chapter_glass}, are basically the same,
except that the glass layer is the top oxide opening and doesn't get any more metal sputtered on it.

An example of the general flow can be seen in \autoref{chapter_silicide_and_cmp} where it already has been performed for the interface area between back end and front end process.

We deposit 150nm LTO and etch holes into it, in order to contact through to the lower layer.

For the first metal layer (\autoref{chapter_metal1}), the etch stop is silicide and we have to sputter Nickel, as a diffusion barrier, before we sputter any Aluminum,
because the $Ti Si_2$ would react with the Aluminum to an high resistivity material.

For simplicity reasons Nickel has been chosen as the passivation material on top of the Aluminum as well, because then we have less different sputter sources.

For the other two layers, \autoref{chapter_metal2} and \autoref{chapter_metal3}, it's only the Aluminum and then the Nickel passivation.

Argon plasma etching with 5\% Chlorine gas added, with an output power of 500 Watts or more has been prooven very effective, for etching Nickel and Aluminium alike.

For the oxide hole etching, any oxide etching machine can be used, or one could even use just buffered oxide etchant (BOE).

\newpage

\subsection{Contacts}\label{chapter_contact}

Now we have to build the first set of vias connecting the first metal layer to the active area.
These vias are in the fringe between front-end and back-end process.

\begin{figure}[H]
	\centering
	\begin{tikzpicture}[node distance = 3cm, auto, thick,scale=\CrossAndTopSectionBig, every node/.style={transform shape}]
		\fill[isolationoxide] (0,0) rectangle (55.0,\LowerMetal);

\input{tikz_process_steps/pimplant.a.tex}

\filldraw[line width=0, nitride] (5.00,\STIIslandSurface) -- (4.50,\STIIslandSurface) -- (5.00,\STIIslandSurface+1.4);
\filldraw[line width=0, nitride] (6.00,\STIIslandSurface) -- (6.50,\STIIslandSurface) -- (6.00,\STIIslandSurface+1.4);

\filldraw[line width=0, nitride] (12.00,\STIIslandSurface) -- (11.50,\STIIslandSurface) -- (12.00,\STIIslandSurface+1.4);
\filldraw[line width=0, nitride] (13.50,\STIIslandSurface) -- (13.00,\STIIslandSurface) -- (13.00,\STIIslandSurface+1.4);

\filldraw[line width=0, nitride] (21.40,\STIIslandSurface) -- (21.90,\STIIslandSurface) -- (21.90,\STIIslandSurface+1.4);
\filldraw[line width=0, nitride] (22.90,\STIIslandSurface) -- (23.40,\STIIslandSurface) -- (22.90,\STIIslandSurface+1.4);


\fill[nitride] (44.0,\polytop+0.75) rectangle (44.8,\polytop+0.95);
\fill[nitride] (44.8,\polytop+0.75) rectangle (45.0,\implantstoptop);
\fill[nitride] (44.8,\implantstoptop) rectangle (46.2,\implantstoptop+0.2);
\fill[nitride] (46.0,\polytop+0.75) rectangle (46.2,\implantstoptop);
\fill[nitride] (46.2,\polytop+0.75) rectangle (47.0,\polytop+0.95);

\fill[silicide] ( 1.35,\STIIslandSurface-0.1) rectangle ( 2.35,\STIIslandSurface);
\fill[silicide] ( 2.70,\STIIslandSurface-0.1) rectangle ( 4.50,\STIIslandSurface);
\fill[silicide] ( 5.00,\STIIslandSurface+1.3) rectangle ( 6.00,\STIIslandSurface+1.4);
\fill[silicide] ( 6.50,\STIIslandSurface-0.1) rectangle ( 8.15,\STIIslandSurface);

\fill[silicide] ( 9.85,\STIIslandSurface-0.1) rectangle (11.50,\STIIslandSurface);
\fill[silicide] (12.00,\STIIslandSurface+1.3) rectangle (13.00,\STIIslandSurface+1.4);
\fill[silicide] (13.50,\STIIslandSurface-0.1) rectangle (15.25,\STIIslandSurface);
\fill[silicide] (15.60,\STIIslandSurface-0.1) rectangle (16.60,\STIIslandSurface);

\fill[silicide] (19.20,\STIIslandSurface-0.1) rectangle (20.20,\STIIslandSurface);
\fill[silicide] (20.60,\STIIslandSurface-0.1) rectangle (21.40,\STIIslandSurface);
\fill[silicide] (21.90,\STIIslandSurface+1.5) rectangle (22.90,\STIIslandSurface+1.6);
\fill[silicide] (23.40,\STIIslandSurface-0.1) rectangle (24.20,\STIIslandSurface);

\fill[silicide] (26.90,\STIIslandSurface-0.1) rectangle (27.90,\STIIslandSurface);
\fill[silicide] (28.35,\STIIslandSurface-0.1) rectangle (29.35,\STIIslandSurface);
\fill[silicide] (29.80,\STIIslandSurface-0.1) rectangle (30.80,\STIIslandSurface);
\fill[silicide] (31.25,\STIIslandSurface-0.1) rectangle (32.15,\STIIslandSurface);
\fill[silicide] (32.60,\STIIslandSurface-0.1) rectangle (33.60,\STIIslandSurface);

\fill[silicide] (35.60,\STIIslandSurface-0.1) rectangle (36.50,\STIIslandSurface);
\fill[silicide] (36.95,\STIIslandSurface-0.1) rectangle (37.85,\STIIslandSurface);
\fill[silicide] (38.30,\STIIslandSurface-0.1) rectangle (39.20,\STIIslandSurface);
\fill[silicide] (39.65,\STIIslandSurface-0.1) rectangle (40.55,\STIIslandSurface);
\fill[silicide] (41.00,\STIIslandSurface-0.1) rectangle (41.90,\STIIslandSurface);

% diode contacts
\fill[silicide] (43.00,\STIIslandSurface+2.0) rectangle (44.00,\STIIslandSurface+2.15);
\fill[silicide] (47.00,\STIIslandSurface+2.0) rectangle (48.00,\STIIslandSurface+2.15);

% resistor contacts
\fill[silicide] (48.50,\STIIslandSurface+2.0) rectangle (49.50,\STIIslandSurface+2.15);
\fill[silicide] (53.50,\STIIslandSurface+2.0) rectangle (54.50,\STIIslandSurface+2.15);

\fill[nitride] (49.5,\polytop+0.75) rectangle (53.5,\polytop+0.95);


\paintcontacts{white}{white}{white}

	\end{tikzpicture}
	\caption{Contact geometry target}
	\label{contact_cross_section}
\end{figure}

As can be seen in \autoref{contact_cross_section}, the goal of this step is purely to deposit a layer of isolation oxide,
get the holes into it, down to the silicide and polyside in order to form wires later on.

We do not wanna etch down anywhere else than the silicide/polycide areas because these function as etch stoppers,
while everywhere else we might etch further than desired with small variations in etching time which might result
in a drastic variation in sheet resistance of the junctions and gate.

\section{Metal 1}\label{metal}

Now we've got to build the first interconnect wires, connecting the contact vias to the "metal1" wires, which will provide a way to contact to them with the via1 contact layout.

\begin{figure}[H]
	\centering
	\begin{tikzpicture}[node distance = 3cm, auto, thick,scale=\CrossAndTopSectionBig, every node/.style={transform shape}]
		\fill[isolationoxide] (0.0,2.0) rectangle (55.0,\LowerMetal);

\paintcontacts{brown}{gray}{brown}

\input{tikz_process_steps/pimplant.a.tex}

\filldraw[line width=0, nitride] (5.00,\STIIslandSurface) -- (4.50,\STIIslandSurface) -- (5.00,\STIIslandSurface+1.4);
\filldraw[line width=0, nitride] (6.00,\STIIslandSurface) -- (6.50,\STIIslandSurface) -- (6.00,\STIIslandSurface+1.4);

\filldraw[line width=0, nitride] (12.00,\STIIslandSurface) -- (11.50,\STIIslandSurface) -- (12.00,\STIIslandSurface+1.4);
\filldraw[line width=0, nitride] (13.50,\STIIslandSurface) -- (13.00,\STIIslandSurface) -- (13.00,\STIIslandSurface+1.4);

\filldraw[line width=0, nitride] (21.40,\STIIslandSurface) -- (21.90,\STIIslandSurface) -- (21.90,\STIIslandSurface+1.4);
\filldraw[line width=0, nitride] (22.90,\STIIslandSurface) -- (23.40,\STIIslandSurface) -- (22.90,\STIIslandSurface+1.4);


\fill[nitride] (44.0,\polytop+0.75) rectangle (44.8,\polytop+0.95);
\fill[nitride] (44.8,\polytop+0.75) rectangle (45.0,\implantstoptop);
\fill[nitride] (44.8,\implantstoptop) rectangle (46.2,\implantstoptop+0.2);
\fill[nitride] (46.0,\polytop+0.75) rectangle (46.2,\implantstoptop);
\fill[nitride] (46.2,\polytop+0.75) rectangle (47.0,\polytop+0.95);

\fill[silicide] ( 1.35,\STIIslandSurface-0.1) rectangle ( 2.35,\STIIslandSurface);
\fill[silicide] ( 2.70,\STIIslandSurface-0.1) rectangle ( 4.50,\STIIslandSurface);
\fill[silicide] ( 5.00,\STIIslandSurface+1.3) rectangle ( 6.00,\STIIslandSurface+1.4);
\fill[silicide] ( 6.50,\STIIslandSurface-0.1) rectangle ( 8.15,\STIIslandSurface);

\fill[silicide] ( 9.85,\STIIslandSurface-0.1) rectangle (11.50,\STIIslandSurface);
\fill[silicide] (12.00,\STIIslandSurface+1.3) rectangle (13.00,\STIIslandSurface+1.4);
\fill[silicide] (13.50,\STIIslandSurface-0.1) rectangle (15.25,\STIIslandSurface);
\fill[silicide] (15.60,\STIIslandSurface-0.1) rectangle (16.60,\STIIslandSurface);

\fill[silicide] (19.20,\STIIslandSurface-0.1) rectangle (20.20,\STIIslandSurface);
\fill[silicide] (20.60,\STIIslandSurface-0.1) rectangle (21.40,\STIIslandSurface);
\fill[silicide] (21.90,\STIIslandSurface+1.5) rectangle (22.90,\STIIslandSurface+1.6);
\fill[silicide] (23.40,\STIIslandSurface-0.1) rectangle (24.20,\STIIslandSurface);

\fill[silicide] (26.90,\STIIslandSurface-0.1) rectangle (27.90,\STIIslandSurface);
\fill[silicide] (28.35,\STIIslandSurface-0.1) rectangle (29.35,\STIIslandSurface);
\fill[silicide] (29.80,\STIIslandSurface-0.1) rectangle (30.80,\STIIslandSurface);
\fill[silicide] (31.25,\STIIslandSurface-0.1) rectangle (32.15,\STIIslandSurface);
\fill[silicide] (32.60,\STIIslandSurface-0.1) rectangle (33.60,\STIIslandSurface);

\fill[silicide] (35.60,\STIIslandSurface-0.1) rectangle (36.50,\STIIslandSurface);
\fill[silicide] (36.95,\STIIslandSurface-0.1) rectangle (37.85,\STIIslandSurface);
\fill[silicide] (38.30,\STIIslandSurface-0.1) rectangle (39.20,\STIIslandSurface);
\fill[silicide] (39.65,\STIIslandSurface-0.1) rectangle (40.55,\STIIslandSurface);
\fill[silicide] (41.00,\STIIslandSurface-0.1) rectangle (41.90,\STIIslandSurface);

% diode contacts
\fill[silicide] (43.00,\STIIslandSurface+2.0) rectangle (44.00,\STIIslandSurface+2.15);
\fill[silicide] (47.00,\STIIslandSurface+2.0) rectangle (48.00,\STIIslandSurface+2.15);

% resistor contacts
\fill[silicide] (48.50,\STIIslandSurface+2.0) rectangle (49.50,\STIIslandSurface+2.15);
\fill[silicide] (53.50,\STIIslandSurface+2.0) rectangle (54.50,\STIIslandSurface+2.15);

\fill[nitride] (49.5,\polytop+0.75) rectangle (53.5,\polytop+0.95);


	\end{tikzpicture}
	\begin{tikzpicture}[node distance = 3cm, auto, thick,scale=\CrossAndTopSectionBig, every node/.style={transform shape}]
		\input{tikz_process_steps/metal.b.tex}
	\end{tikzpicture}
	\caption{Metal geometry target}
	\label{metal_target}
\end{figure}

As can be seen in \autoref{metal_target}, the goal of this step is purely to etch the wire structure for the first metal layer into the in \autoref{metal_deposition} deposited metal layer, and form wires by doing so.

\begin{figure}[H]
	\centering
	\begin{tikzpicture}[node distance =1cm, auto, thick,scale=\VLSILayout, every node/.style={transform shape}]
		\input{tikz_process_steps/metal.layout.tex}
	\end{tikzpicture}
	\caption{First metal layout}
	\label{metal_layout}
\end{figure}

It should be noted again that the via placement and dimensions in \autoref{metal_layout} are solely for demonstration purposes for the process and are in no way the actual standard cell design for the final standard cell lib. \\

In later iterations of this process we might be switching to Tungsten as the metal material for this step so the etching method might change in further releases.

\newpage

\subsection{Metal deposition}\label{metal_deposition}

Now we somehow have got to get the metal onto our silicon oxide in a fashion so that it fills the holes we've etched in \autoref{contact_holes_etch} and touches down onto the silicide/polycide, thus making a contact to the active area.

\begin{figure}[H]
	\centering
	\begin{tikzpicture}[node distance = 3cm, auto, thick,scale=\CrossSectionOnly, every node/.style={transform shape}]
		\input{tikz_process_steps/metal.metal_deposition.a.tex}
	\end{tikzpicture}
	\drawStepArrow{}
	\begin{tikzpicture}[node distance = 3cm, auto, thick,scale=\CrossSectionOnly, every node/.style={transform shape}]
		\input{tikz_process_steps/metal.metal_deposition.b.tex}
	\end{tikzpicture}
	\caption{Metal deposition}
\end{figure}

In order to reach the target of filling the holes in the oxide and having at least another depth worth of space in order to have an enough low resistance of the wire interconnect.
We end up with a target thickness of 4\um.

\textbf{Possible approaches}:
\begin{itemize}
	\item \textbf{"Varian 3180 Sputter (SPT-3180)" from HKUST} \\
	The deposition speed is 16nm/s which gives us a required deposition time of 250 seconds for 4\um.
	\item \textbf{Add your solution here!}
\end{itemize}

\newpage

\subsection{Etching}\label{metal_wire_etch}

Now we've got to etch the Aluminum which hasn't been covered yet by the resist in order to get the desired wire structures.

\begin{figure}[H]
	\centering
	\begin{tikzpicture}[node distance = 3cm, auto, thick,scale=\CrossAndTopSection, every node/.style={transform shape}]
		\input{tikz_process_steps/metal.etching.a.tex}
	\end{tikzpicture}
	\begin{tikzpicture}[node distance = 3cm, auto, thick,scale=\CrossAndTopSection, every node/.style={transform shape}]
		\input{tikz_process_steps/metal.etching.at.tex}
	\end{tikzpicture}
	\drawStepArrow{Mask: metal1}
	\begin{tikzpicture}[node distance = 3cm, auto, thick,scale=\CrossAndTopSection, every node/.style={transform shape}]
		\input{tikz_process_steps/metal.etching.b.tex}
	\end{tikzpicture}
	\begin{tikzpicture}[node distance = 3cm, auto, thick,scale=\CrossAndTopSection, every node/.style={transform shape}]
		\input{tikz_process_steps/metal.etching.bt.tex}
	\end{tikzpicture}
	\caption{Etching first wires}
\end{figure}

\textbf{Possible approaches}:
\begin{itemize}
	\item \textbf{"Oxford Aluminum Etcher (DRY-Metal-2)" from HKUST} \\
	The normal etch rate for Aluminum is 180 nm/min with this machines \\
	We've deposited 4\um Aluminum in \autoref{metal_deposition} which means we've got to etch for around 22 minutes and 13 seconds
	\item \textbf{Chemical solution} \\
	Please specify here!
\end{itemize}


\newpage

\subsection{Via 1}\label{chapter_via1}

Now we have to build an additional set of via1. connecting the first metal layer to the next metal layer.
These via1. are already part of the front-end process.

\begin{figure}[H]
	\centering
	\begin{tikzpicture}[node distance = 3cm, auto, thick,scale=\CrossSectionOnly, every node/.style={transform shape}]
		\fill[isolationoxide] (0,\LowerMetal) rectangle (55,\LowerMoreMetal);

\paintscaledvias{white}{\UpperMetal}{\LowerMoreMetal}{0}

\fill[nitride] (0.0,\LowerMetal) rectangle (55.0,\LowerMetal+0.5);
\fill[isolationoxide] (0.0,2.0) rectangle (55.0,\LowerMetal+0.25);

\paintscaledmetal{nitride}{0.25}{0.5}
\paintscaledmetal{isolationoxide}{0.0}{0.25}

\input{tikz_process_steps/silicification.a.tex}

\paintcontacts{brown}{gray}{brown}


	\end{tikzpicture}
	\caption{Contact geometry target}
	\label{via1.cross_subsections}
\end{figure}

As can be seen in \autoref{via1.cross_subsections}, the goal of this step is purely to deposit a layer of isolation oxide, get the holes into it, down to the metal layer below in order to form wires later on.

In a later iterations of this process we might be switching to Copper as the metal material for this step which will result in a variation of this step because the usage of damascene method.

\section{Metal 2}\label{more_metal}
Now we've got to build the more interconnect wires, connecting the contact vias to the "metal2" wires, which will provide a way to contact to them with the via2 contact layout.

\begin{figure}[H]
	\centering
	\begin{tikzpicture}[node distance = 3cm, auto, thick,scale=\CrossAndTopSectionBig, every node/.style={transform shape}]
		\fill[isolationoxide] (0,\LowerMetal) rectangle (55,\LowerMoreMetal);

\paintscaledvias{white}{\UpperMetal}{\LowerMoreMetal}{0}

\input{tikz_process_steps/metal.a.tex}

\paintscaledvias{metal2}{\UpperMetal}{\LowerMoreMetal}{0.0}
\paintscaledvias{metal2}{\LowerMoreMetal}{\UpperMoreMetal}{0.25}

	\end{tikzpicture}
	\caption{Metal geometry target}
	\label{more_metal_target}
\end{figure}

As can be seen in \autoref{more_metal_target}, the goal of this step is purely to etch the wire structure for the additional metal layer into the in \autoref{more_metal_deposition} deposited metal layer, and form wires by doing so.

In later iterations of this process we might be switching to Tungsten as the metal material for this step so the etching method might change in further releases.

\newpage

\subsection{Metal deposition}\label{more_metal_deposition}

Now we somehow have got to get the metal onto our silicon oxide in a fashion so that it fills the holes we've etched in \autoref{via_etching} and touches down onto the last metal layer, thus making a contact to the plane below.

\begin{figure}[H]
	\centering
	\begin{tikzpicture}[node distance = 3cm, auto, thick,scale=\CrossSectionOnly, every node/.style={transform shape}]
		\input{tikz_process_steps/more_metal.metal_deposition.a.tex}
	\end{tikzpicture}
	\drawStepArrow{}
	\begin{tikzpicture}[node distance = 3cm, auto, thick,scale=\CrossSectionOnly, every node/.style={transform shape}]
		\input{tikz_process_steps/more_metal.metal_deposition.b.tex}
	\end{tikzpicture}
	\caption{Metal deposition}
\end{figure}

In order to reach the target of filling the holes in the oxide and having at least another depth worth of space in order to have an enough low resistance of the wire interconnect.
We end up with a target thickness of 4\um.

\textbf{Possible approaches}:
\begin{itemize}
	\item \textbf{"Varian 3180 Sputter (SPT-3180)" from HKUST} \\
	The deposition speed is 16nm/s which gives us a required deposition time of 250 seconds for 4\um.
	\item \textbf{Add your solution here!}
\end{itemize}

\newpage

\subsection{Etching}\label{more_metal_wire_etch}

Now we've got to etch the Aluminum which hasn't been covered yet by the resist in order to get the desired wire structures.

\begin{figure}[H]
	\centering
	\begin{tikzpicture}[node distance = 3cm, auto, thick,scale=\CrossAndTopSection, every node/.style={transform shape}]
		\input{tikz_process_steps/more_metal.etching.a.tex}
	\end{tikzpicture}
	\begin{tikzpicture}[node distance = 3cm, auto, thick,scale=\CrossAndTopSection, every node/.style={transform shape}]
		\input{tikz_process_steps/more_metal.etching.at.tex}
	\end{tikzpicture}
	\drawStepArrow{Mask: metal1}
	\begin{tikzpicture}[node distance = 3cm, auto, thick,scale=\CrossAndTopSection, every node/.style={transform shape}]
		\input{tikz_process_steps/more_metal.etching.b.tex}
	\end{tikzpicture}
	\begin{tikzpicture}[node distance = 3cm, auto, thick,scale=\CrossAndTopSection, every node/.style={transform shape}]
		\input{tikz_process_steps/more_metal.etching.bt.tex}
	\end{tikzpicture}
	\caption{Etching first wires}
\end{figure}

\textbf{Possible approaches}:
\begin{itemize}
	\item \textbf{"Oxford Aluminum Etcher (DRY-Metal-2)" from HKUST} \\
	The normal etch rate for Aluminum is 180 nm/min with this machines \\
	We've deposited 4\um Aluminum in \autoref{more_metal_deposition} which means we've got to etch for around 22 minutes and 13 seconds
	\item \textbf{Chemical solution} \\
	Please specify here!
\end{itemize}


\newpage

\subsection{Via 2}\label{chapter_via2}

Now we have to build an additional set of via1. connecting the first metal layer to the next metal layer.
These via1. are already part of the front-end process.

\begin{figure}[H]
	\centering
	\begin{tikzpicture}[node distance = 3cm, auto, thick,scale=\CrossSectionOnly, every node/.style={transform shape}]
		\fill[isolationoxide] (0,\LowerMetal) rectangle (55,\LowerMoreMetal);

\paintscaledvias{white}{\UpperMetal}{\LowerMoreMetal}{0}

\fill[nitride] (0.0,\LowerMetal) rectangle (55.0,\LowerMetal+0.5);
\fill[isolationoxide] (0.0,2.0) rectangle (55.0,\LowerMetal+0.25);

\paintscaledmetal{nitride}{0.25}{0.5}
\paintscaledmetal{isolationoxide}{0.0}{0.25}

\input{tikz_process_steps/silicification.a.tex}

\paintcontacts{brown}{gray}{brown}


	\end{tikzpicture}
	\caption{Contact geometry target}
	\label{via1.cross_subsections}
\end{figure}

As can be seen in \autoref{via1.cross_subsections}, the goal of this step is purely to deposit a layer of isolation oxide, get the holes into it, down to the metal layer below in order to form wires later on.

In a later iterations of this process we might be switching to Copper as the metal material for this step which will result in a variation of this step because the usage of damascene method.

\newpage

\subsubsection{Isolation dioxide layer}

We now need to grow a layer of thick oxide in order to isolate the Aluminum interconnect layer from the active area.

\begin{figure}[H]
	\centering
	\begin{tikzpicture}[node distance = 3cm, auto, thick,scale=\CrossSectionOnly, every node/.style={transform shape}]
		\fill[nitride] (0.0,\LowerMetal) rectangle (55.0,\LowerMetal+0.5);
\fill[isolationoxide] (0.0,2.0) rectangle (55.0,\LowerMetal+0.25);

\paintscaledmetal{nitride}{0.25}{0.5}
\paintscaledmetal{isolationoxide}{0.0}{0.25}

\input{tikz_process_steps/silicification.a.tex}

\paintcontacts{brown}{gray}{brown}


	\end{tikzpicture}
	\drawStepArrow{LTO\\deposition}
	\begin{tikzpicture}[node distance = 3cm, auto, thick,scale=\CrossSectionOnly, every node/.style={transform shape}]
		\fill[isolationoxide] (0,\LowerMetal) rectangle (20,\LowerMoreMetal);
\fill[nitride] (0.0,\LowerMetal) rectangle (55.0,\LowerMetal+0.5);
\fill[isolationoxide] (0.0,2.0) rectangle (55.0,\LowerMetal+0.25);

\paintscaledmetal{nitride}{0.25}{0.5}
\paintscaledmetal{isolationoxide}{0.0}{0.25}

\input{tikz_process_steps/silicification.a.tex}

\paintcontacts{brown}{gray}{brown}


	\end{tikzpicture}
	\caption{Oxide layer}
\end{figure}

\textbf{Possible approaches}:
\begin{itemize}
	\item \textbf{"LPCVD-B3 LTO (CVD-B3)" from HKUST} \\
	At HKUST we have a chemical vapor deposition unit which gives us better control over the layer thicknes. \\
	These steps are needed to arrive with the desired geometry\footnote{\url{http://memslab.blogspot.com/2013/01/lto-lpcvd.html}}
	\begin{enumerate}
		\item Set the growth rate to 14 nm/min
		\item Run for 140 minutes
	\end{enumerate}
	\item \textbf{In a furnace ("a hack around")} \\
	In case of a lack of LPCVD equipment one might also resort to "hack together" a solution for LTO deposition using a furnace\footnote{\url{https://www.sciencedirect.com/science/article/pii/0167577X89900062}}
		\begin{enumerate}
			\item Deposit tetraethyl orthosilicate ($Si C_8 H_{20} O_4$)
			\item React for 20 minutes at 1050\degreesC in $N_2$ environment in a furnace
	\end{enumerate}
\end{itemize}

\newpage

\subsubsection{Etching}\label{via1.etching}

We now need to open a window in the dioxide layer, through which we will inject carrier atoms into the silicon crystal structure.

\begin{figure}[H]
	\centering
	\begin{tikzpicture}[node distance = 3cm, auto, thick,scale=\CrossAndTopSection, every node/.style={transform shape}]
		\fill[isolationoxide] (0,\LowerMetal) rectangle (20,\LowerMoreMetal);

\fill[nitride] (0.0,\LowerMetal) rectangle (55.0,\LowerMetal+0.5);
\fill[isolationoxide] (0.0,2.0) rectangle (55.0,\LowerMetal+0.25);

\paintscaledmetal{nitride}{0.25}{0.5}
\paintscaledmetal{isolationoxide}{0.0}{0.25}

\input{tikz_process_steps/silicification.a.tex}

\paintcontacts{brown}{gray}{brown}


\fill[resist] (0,\LowerMoreMetal) rectangle (1,\UpperMoreMetalResist);
\fill[resist] (3,\LowerMoreMetal) rectangle (5,\UpperMoreMetalResist);
\fill[resist] (6.5,\LowerMoreMetal) rectangle (9,\UpperMoreMetalResist);
\fill[resist] (11,\LowerMoreMetal) rectangle (13.5,\UpperMoreMetalResist);
\fill[resist] (15.0,\LowerMoreMetal) rectangle (17.0,\UpperMoreMetalResist);
\fill[resist] (19,\LowerMoreMetal) rectangle (20,\UpperMoreMetalResist);


	\end{tikzpicture}
	\begin{tikzpicture}[node distance = 3cm, auto, thick,scale=\CrossAndTopSection, every node/.style={transform shape}]
		\input{tikz_process_steps/via1.etching.at.tex}
	\end{tikzpicture}
	\drawStepArrow{}
	\begin{tikzpicture}[node distance = 3cm, auto, thick,scale=\CrossAndTopSection, every node/.style={transform shape}]
		\fill[isolationoxide] (0,\LowerMetal) rectangle (1,\LowerMoreMetal);
\fill[isolationoxide] (3,\LowerMetal) rectangle (5,\LowerMoreMetal);
\fill[isolationoxide] (6.5,\LowerMetal) rectangle (9,\LowerMoreMetal);
\fill[isolationoxide] (11,\LowerMetal) rectangle (13.5,\LowerMoreMetal);
\fill[isolationoxide] (15.0,\LowerMetal) rectangle (17.0,\LowerMoreMetal);
\fill[isolationoxide] (19.0,\LowerMetal) rectangle (20.0,\LowerMoreMetal);

\fill[nitride] (0.0,\LowerMetal) rectangle (55.0,\LowerMetal+0.5);
\fill[isolationoxide] (0.0,2.0) rectangle (55.0,\LowerMetal+0.25);

\paintscaledmetal{nitride}{0.25}{0.5}
\paintscaledmetal{isolationoxide}{0.0}{0.25}

\input{tikz_process_steps/silicification.a.tex}

\paintcontacts{brown}{gray}{brown}


\fill[resist] (0,\LowerMoreMetal) rectangle (1,\UpperMoreMetalResist);
\fill[resist] (3,\LowerMoreMetal) rectangle (5,\UpperMoreMetalResist);
\fill[resist] (6.5,\LowerMoreMetal) rectangle (9,\UpperMoreMetalResist);
\fill[resist] (11,\LowerMoreMetal) rectangle (13.5,\UpperMoreMetalResist);
\fill[resist] (15.0,\LowerMoreMetal) rectangle (17.0,\UpperMoreMetalResist);
\fill[resist] (19,\LowerMoreMetal) rectangle (20,\UpperMoreMetalResist);


	\end{tikzpicture}
	\begin{tikzpicture}[node distance = 3cm, auto, thick,scale=\CrossAndTopSection, every node/.style={transform shape}]
		\input{tikz_process_steps/via1.etching.bt.tex}
	\end{tikzpicture}
	\caption{N+ region opened}
\end{figure}

Since the silicon dioxide layer is 100nm thick and we wanna reach the silicon below we can use wet etching as described in the chemistry chapter.\\

\textbf{Possible approaches}:
\begin{itemize}
	\item \textbf{"AOE Etcher (DRY-AOE)" from HKUST} \\
	We can use anisotropic plasma etching for sharper borders.
	\item \textbf{Chemical solution} \\
	We can use buffered hydrofluoric acid (BOE (1:6)) at room temperature ($\approx$508 nm/min) for around 4 minutes in order to get through the 2\um of oxide.\\
	Too long over 4 minutes might cause under-etch however!
\end{itemize}

\subsection{Metal 3}\label{chapter_metal3}
Now we've got to build the more interconnect wires, connecting the contact vias to the "metal2" wires, which will provide a way to contact to them with the via2 contact layout.

\begin{figure}[H]
	\centering
	\begin{tikzpicture}[node distance = 3cm, auto, thick,scale=\CrossAndTopSectionBig, every node/.style={transform shape}]
		\fill[isolationoxide] (0.0,\LowerMoreMetal) rectangle (55.0,\LowerMoreMetalTwo);

\input{tikz_process_steps/metal2.a.tex}

\paintscaledvias{white}{\UpperMoreMetal}{\LowerMoreMetalTwo}{0}


\fill[nitride] (0,\LowerMoreMetalTwo) rectangle (55,\LowerMoreMetalTwo+0.5);
\fill[isolationoxide] (0,\LowerMoreMetalTwo) rectangle (55,\LowerMoreMetalTwo+0.25);

\paintscaledvias{nitride}{\LowerMoreMetalTwo+0.25}{\UpperMoreMetalTwo+0.5}{1.00}
\paintscaledvias{isolationoxide}{\LowerMoreMetalTwo}{\UpperMoreMetalTwo+0.25}{0.75}

\paintscaledvias{gray}{\UpperMoreMetal}{\LowerMoreMetalTwo}{0.25}
\paintscaledvias{gray}{\LowerMoreMetalTwo}{\UpperMoreMetalTwo}{0.50}

\paintscaledvias{brown}{\UpperMoreMetal+0.25}{\LowerMoreMetalTwo+0.3}{-0.1}
\paintscaledvias{brown}{\LowerMoreMetalTwo+0.3}{\UpperMoreMetalTwo}{0.5}

	\end{tikzpicture}
	\caption{Metal geometry target}
	\label{metal3_target}
\end{figure}

As can be seen in \autoref{metal3_target}, the goal of this step is purely to etch the wire structure for the additional metal layer into the in \autoref{metal3_deposition} deposited metal layer, and form wires by doing so.

In later iterations of this process we might be switching to Tungsten as the metal material for this step so the etching method might change in further releases.



\newpage

\subsection{Glass}\label{chapter_glass}

This is the final oxide layer, which serves as a passivation for the metal3 wires and exposes the bonding and test pads
to the outside world.

\begin{figure}[H]
	\centering
	\begin{tikzpicture}[node distance = 3cm, auto, thick,scale=\CrossSectionOnly, every node/.style={transform shape}]
		\fill[isolationoxide] (0.0,\LowerMoreMetalTwo) rectangle (55.0,\UpperGlass);

\input{tikz_process_steps/via2.a.tex}

\fill[nitride] (0,\LowerMoreMetalTwo) rectangle (55,\LowerMoreMetalTwo+0.5);
\fill[isolationoxide] (0,\LowerMoreMetalTwo) rectangle (55,\LowerMoreMetalTwo+0.25);

\paintscaledvias{nitride}{\LowerMoreMetalTwo+0.25}{\UpperMoreMetalTwo+0.5}{1.00}
\paintscaledvias{isolationoxide}{\LowerMoreMetalTwo}{\UpperMoreMetalTwo+0.25}{0.75}

\paintscaledvias{gray}{\UpperMoreMetal}{\LowerMoreMetalTwo}{0.25}
\paintscaledvias{gray}{\LowerMoreMetalTwo}{\UpperMoreMetalTwo}{0.50}

\paintscaledvias{brown}{\UpperMoreMetal+0.25}{\LowerMoreMetalTwo+0.3}{-0.1}
\paintscaledvias{brown}{\LowerMoreMetalTwo+0.3}{\UpperMoreMetalTwo}{0.5}


\paintscaledvias{white}{\UpperMoreMetalTwo}{\UpperGlass}{0.25}

	\end{tikzpicture}
	\caption{Glass geometry target}
	\label{glass_cross_sections}
\end{figure}

As can be seen in \autoref{glass_cross_sections}, the goal of this step is purely to deposit a layer of isolation oxide,
get the holes into it, down to the metal layer below in order to form wires later on.

In a later iterations of this process we might be switching to Copper as the metal material for this step which will
result in a variation of this step because the usage of damascene method.


