\section{Junction implants}\label{junctions_chapter}

After we've etches the gate structures and have implemented the implant stop structures we perform the junction implants in \autoref{pimplant_chapter} and \autoref{nimplant_chapter}.

Thanks to the implant stop mask we have a nice control over channel lenghts because the implant stop structure compensates for alignment issues, which is crucial for Polysilicon diodes for instance.

\begin{figure}[H]
	\centering
	\begin{tikzpicture}[node distance = 3cm, auto, thick,scale=\CrossAndTopSectionBig, every node/.style={transform shape}]
		\input{tikz_process_steps/fox.a.tex}

\paintnwells{3.5}
\paintpwells{3.5}
\paintpbases{2.0}
\paintnbases{1.5}

\fill[gateoxide] (5.00,\STIIslandSurface) rectangle (6.00,\gateoxidetop);
\fill[poly] (5.00,\gateoxidetop) rectangle (6.00,\polytop);

\fill[gateoxide] (12.00,\STIIslandSurface) rectangle (13.00,\gateoxidetop);
\fill[poly] (12.00,\gateoxidetop) rectangle (13.00,\polytop);

\fill[gateoxide] (21.90,\STIIslandSurface) rectangle (22.90,\SONOStopONE);
\fill[nitride] (21.90,\SONOStopONE) rectangle (22.90,\SONOStopTWO);
\fill[gateoxide] (21.90,\SONOStopTWO) rectangle (22.90,\SONOStopTHREE);
\fill[poly] (21.90,\SONOStopTHREE) rectangle (22.90,\polytop+0.2);

% poly diode
\fill[poly] (43.00,\STIIslandSurface+0.75) rectangle (48.00,\polytop+0.75);

% poly resistor 
\fill[poly] (48.50,\STIIslandSurface+0.75) rectangle (54.50,\polytop+0.75);

\fill[isolationoxide] (45.0,\polytop+0.75) rectangle (46.0,\implantstoptop);


\paintnimplants{0.5}
\paintpimplants{0.5}


	\end{tikzpicture}
	\caption{P+ and N+ junctions}
\end{figure}

After implantation we have to drive in the shallow junction implants in order to form the pn-junctions in the polysilicon gates and to increase the depth of the junctions
to a degree so that it won't be fully consumed during the silicide formation(\autoref{step_silicification}) step later on.

After we've implanted the Boron and Phosphorus, we will drive the whole thing in for 30 minutes at 900\degreesC and at the same time oxidize it with $O_2$ (dry oxidation) so that we can form the pad
oxide needed to deposit the nitride for the side wall spacers later on in \autoref{nitride_spacers_deposition}.

\newpage
\subsection{n+ Implant}\label{nimplant_chapter}

For the bulk of the PMOS transistors and for the source and drain of the NMOS transistors highly doped  n+ areas are required.

In this step we're going to build these.

\begin{figure}[H]
	\centering
	\begin{tikzpicture}[node distance = 3cm, auto, thick,scale=\CrossAndTopSectionBig, every node/.style={transform shape}]
		\input{tikz_process_steps/gate.a.tex}
\fill[isolationoxide] (45.0,\polytop+0.75) rectangle (46.0,\implantstoptop);


\paintnimplants{0.25}


	\end{tikzpicture}
	\begin{tikzpicture}[node distance = 3cm, auto, thick,scale=\CrossAndTopSectionBig, every node/.style={transform shape}]
		\input{tikz_process_steps/nimplant.b.tex}
	\end{tikzpicture}
	\caption{N+ implant geometry target}
\end{figure}

The tricky thing here is to have a reasonable implant depth but not too deep because the deeper the junction, the higher the junction capacity which in turn limits the switching performance of the CMOS circuitry.

\begin{figure}[H]
	\centering
	\begin{tikzpicture}[node distance =1cm, auto, thick,scale=\VLSILayout, every node/.style={transform shape}]
		\input{tikz_process_steps/nimplant.layout.tex}
	\end{tikzpicture}
	\caption{N+ layout}
	\label{nimplant_layout}
\end{figure}

An example layout of p-implants can be seen in \autoref{nimplant_layout}, the mask is being extracted from the layer "n\_plus\_select".

The nselect is implanted with a Phosphorus ($P^{31}$) dose of $2.5\times10^{12}cm^{-2}$ at an energy of 30 keV (43nm$\pm$18nm deep)

Also important to notice is that this example layout is just for demonstration purposes only, please have a look at the standard cell documentation for the actual layouts. 

\newpage

\subsection{p+ Implant}\label{pimplant_chapter}
For the bulk of the NMOS transistors and for the source and drain of the PMOS transistors highly doped  p+ areas are required.
In this step we're going to build these.

\begin{figure}[H]
	\centering
	\begin{tikzpicture}[node distance = 3cm, auto, thick,scale=\CrossAndTopSectionBig, every node/.style={transform shape}]
		\input{tikz_process_steps/gate.a.tex}
\fill[isolationoxide] (45.0,\polytop+0.75) rectangle (46.0,\implantstoptop);


\paintnimplants{0.5}
\paintpimplants{0.5}


	\end{tikzpicture}
	\caption{P+ implant geometry target}
\end{figure}

The tricky thing here is to have a reasonable implant depth but not too deep because the deeper the junction, the higher the junction capacity which in turn limits the switching performance of the CMOS circuitry.

Also important to notice is that the implantation energy must not be too high, otherwise the dopants may leak through the polysilicon gate.

The pselect is implanted with a Boron ($B^{11}$) dose of $2.5\times10^{12}cm^{-2}$ at an energy of 20 keV  (43nm$\pm$18nm deep)


