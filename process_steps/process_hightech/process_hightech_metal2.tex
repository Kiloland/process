\subsection{Metal 2}\label{chapter_metal2}

Now we've got to build the more interconnect wires, connecting the "metal1" to the "metal2" wires,
which will provide a way to contact to them with the via2 contact layout.

\begin{figure}[H]
	\centering
	\begin{tikzpicture}[node distance = 3cm, auto, thick,scale=\CrossAndTopSectionBig, every node/.style={transform shape}]
		\fill[isolationoxide] (0,\LowerMetal) rectangle (55,\LowerMoreMetal);

\paintscaledvias{white}{\UpperMetal}{\LowerMoreMetal}{0}

\fill[nitride] (0.0,\LowerMetal) rectangle (55.0,\LowerMetal+0.5);
\fill[isolationoxide] (0.0,2.0) rectangle (55.0,\LowerMetal+0.25);

\paintscaledmetal{nitride}{0.25}{0.5}
\paintscaledmetal{isolationoxide}{0.0}{0.25}

\input{tikz_process_steps/silicification.a.tex}

\paintcontacts{brown}{gray}{brown}


\paintscaledvias{metal2}{\UpperMetal}{\LowerMoreMetal}{0.0}
\paintscaledvias{metal2}{\LowerMoreMetal}{\UpperMoreMetal}{0.25}

	\end{tikzpicture}
	\caption{Metal geometry target}
	\label{metal2_target}
\end{figure}

As can be seen in \autoref{metal2_target}, the goal of this step is purely to etch the wire structure for the additional
metal layer into the in \autoref{metal1_target} deposited metal layer, and form wires by doing so.

In a later iterations of this process we might be switching to Copper as the metal2 material for this step which
will result in a variation of this step because the usage of damascene method.

For now first we sputter 100nm Aluminum and then around 50nm Nickel for passivation, all in the same vacuum.
